\chapter{Dokumentace}

\section{REST API}



\begin{sidewaystable}[h]
\fontsize{8.3}{10}\selectfont
\centering
\begin{tabular}{|l|l|}
\hline
\verb|GET    /|                                          & Vrací schéma API ve formátu OpenAPI \\ \hline
\verb|GET    /builds|                                    & Vrací seznam integrací \\ \hline
\verb|GET    /builds/[build-id]|                         & Vrací integraci s identifikátorem \verb|[build-id]| \\ \hline
\verb|POST   /builds/[build-id]/cancel|                  & Přeruší probíhající integraci s identifikátorem \verb|[build-id]| \\ \hline
\verb|POST   /builds/[build-id]/cancel|                  & Restartuje integraci s identifikátorem \verb|[build-id]| \\ \hline
\verb|GET    /builds/[build-id]/stages|                  & Vrací seznam fází integrace s identifikátorem \verb|[build-id]| \\ \hline
\verb|GET    /jobs|                                      & Vrací seznam úkolů \\ \hline
\verb|GET    /jobs/[job-id]|                             & Vrací úkol s identifikátorem \verb|[job-id]| \\ \hline
\verb|POST   /jobs/[job-id]/cancel|                      & Přeruší probíhající úkol s identifikátorem \verb|[job-id]| \\ \hline
\verb|POST   /jobs/[job-id]/cancel|                      & Restartuje úkol s identifikátorem \verb|[job-id]| \\ \hline
\verb|GET    /jobs/queue/[runner-token]|                 & Vrací úkol ke zpracování běhovým klientem identifikovaným \verb|[runner-token]| \\ \hline
\verb|POST   /jobs/report/[job-secret]|                  & Přijímá data od běhového klienta o probíhajícím úkolu, obsahuje stav úkolu a log \\ \hline
\verb|GET    /jobs/[job-id]/log|                         & Vrací log úkolu \\ \hline
\verb|GET    /projects|                                  & Vrací seznam projektů \\ \hline
\verb|POST   /projects|                                  & Vytváří nový projekt \\ \hline
\verb|GET    /projects/[project-id]|                     & Vrací projekt s identifikátorem \verb|[project-id]| \\ \hline
\verb|PUT    /projects/[project-id]|                     & Editace projektu s identifikátorem \verb|[project-id]| \\ \hline
\verb|DELETE /projects/[project-id]|                     & Smazání projektu s identifikátorem \verb|[project-id]| \\ \hline
\verb|GET    /projects/[project-id]/builds|              & Vrací seznam integrací spojených s projektem identikovaným \verb|[project-id]| \\ \hline
\verb|GET    /projects/[project-id]/stages|              & Vrací seznam fází spojených s projektem identikovaným \verb|[project-id]| \\ \hline
\verb|GET    /projects/[project-id]/users|               & Vrací seznam uživatelů přiřazených do projektu identikovaným \verb|[project-id]| \\ \hline
\verb|POST   /projects/[project-id]/users|               & Přidává uživatele do projektu identikovaným \verb|[project-id]| \\ \hline
\verb|DELETE /projects/[project-id]/users/[user-id]|     & Odstraňuje uživatele z projektu identikovaným \verb|[project-id]| \\ \hline
\verb|GET    /stages|                                    & Vrací seznam fází \\ \hline
\verb|GET    /stages/[stage-id]|                         & Vrací fázi s identifikátorem \verb|[stage-id]| \\ \hline
\verb|POST   /stages/[stage-id]/cancel|                  & Přeruší probíhající fázi s identifikátorem \verb|[stage-id]| \\ \hline
\verb|POST   /stages/[stage-id]/cancel|                  & Restartuje fázi s identifikátorem \verb|[stage-id]| \\ \hline
\verb|GET    /runners|                                   & Vrací seznam běhových klientů \\ \hline
\verb|POST   /runners|                                   & Vytváří nového běhového klienta \\ \hline
\verb|GET    /runners/[runner-id]|                       & Vrací běhového klienta s identifikátorem \verb|[runner-id]| \\ \hline
\verb|PUT    /runners/[runner-id]|                       & Edituje běhového klienta s identifikátorem \verb|[runner-id]| \\ \hline
\verb|DELETE /runners/[runner-id]|                       & Maže běhového klienta s identifikátorem \verb|[runner-id]| \\ \hline
\verb|GET    /users|                                     & Vrací seznam uživatelů \\ \hline
\verb|POST   /users|                                     & Vytváří nového uživatele \\ \hline
\verb|GET    /users/[user-id]|                           & Vrací uživatele s identifikátorem \verb|[user-id]| \\ \hline
\verb|PUT    /users/[user-id]|                           & Edituje uživatele s identifikátorem \verb|[user-id]| \\ \hline
\verb|DELETE /users/[user-id]|                           & Maže uživatele s identifikátorem \verb|[user-id]| \\ \hline
\verb|GET    /identity|                                  & Vrací aktuální identitu na základě autentizačního tokenu \\ \hline
\end{tabular}
\end{sidewaystable}

\clearpage\section{Terminálové rozhraní}
{

\setminted{fontsize=\scriptsize,baselinestretch=1}

\begin{listing}[H]
\begin{minted}[frame=single]{text}
$ help

Documented commands (type help <topic>):
========================================
build  exit  help  identity  job  project  runner  stage  user
\end{minted}
\end{listing}

\begin{listing}[H]
\begin{minted}[frame=single]{text}
$ help identity

> identity get
> identity update [email = str] [role = (master, admin, normal)] 
                  [token = str] [public_key = str]
\end{minted}
\end{listing}

\begin{listing}[H]
\begin{minted}[frame=single]{text}
$ help identity

> identity get
> identity update [email = str] [role = (master, admin, normal)] 
                  [token = str] [public_key = str]
\end{minted}
\end{listing}

\begin{listing}[H]
\begin{minted}[frame=single]{text}
$ help project

> project get [project_id]
> project list [url = str] [origin = str]
  [limit = int] [offset = int] [order = (created-desc|created-asc)]
> project count [url = str] [origin = str]
> project update [project_id] [url = str] [origin = str]
> project create [url = str] [origin = str]
\end{minted}
\end{listing}

\begin{listing}[H]
\begin{minted}[frame=single]{text}
$ help build

> build get [build_id]
> build list [project_id = int] [branch = str]
  [status = (created|ready|pending|running|failed|success|canceled|skipped|error)]
  [limit = int] [offset = int] [order = (created-desc|created-asc)]
> build count [project_id = int] [branch = str] [status = ()]
  [limit = int] [offset = int] [order = (created-desc|created-asc)]
> build cancel [build_id]
> build restart [build_id]
\end{minted}
\end{listing}

\begin{listing}[H]
\begin{minted}[frame=single]{text}
$ help stage

> stage get [stage_id]
> stage list [project_id = int] [build_id = int]
  [status = (created|ready|pending|running|failed|success|canceled|skipped|error)]
  [limit = int] [offset = int]
> stage count
  [project_id = int] [build_id = int]
  [status = (created|ready|pending|running|failed|success|canceled|skipped|error)]
> stage cancel [stage_id]
> stage restart [stage_id]
\end{minted}
\end{listing}

\begin{listing}[H]
\begin{minted}[frame=single]{text}
$ help job

> job get [job_id]
> job list
  [project_id = int] [build_id = int] [stage_id = int]
  [status = (created|ready|pending|running|failed|success|canceled|skipped|error)]
  [limit = int] [offset = int]
> job count
  [project_id = int] [build_id = int] [stage_id = int]
  [status = (created|ready|pending|running|failed|success|canceled|skipped|error)]
> job cancel [job_id]
> job restart [job_id]
> job log [job_id]
\end{minted}
\end{listing}

\begin{listing}[H]
\begin{minted}[frame=single]{text}
$ help user

> user get [user_id]
> user list [email = str]
  [limit = int] [offset = int] [order = (created-desc|created-asc)]
> user count [email = str]
  [limit = int] [offset = int] [order = (created-desc|created-asc)]
> user create [email = str] [role = (master, admin, normal)] [token = str]
  [public_key = str]
> user update [email = str] [role = (master, admin, normal)] [token = str]
  [public_key = str]
> user delete [user_id]
\end{minted}
\end{listing}

\begin{listing}[H]
\begin{minted}[frame=single]{text}
$ help runner

> runner get [runner_id]
> runner list [group = str]
  [limit = int] [offset = int]
> runner count [group = str]
> runner create [group = str] [token = str]
> runner update [group = str] [token = str]
> runner delete [runner_id]
\end{minted}
\end{listing}
}

\clearpage\section{Postup nasazení}

\subsection{Piper CI core}

Požadavky na systém:

\begin{itemize}
	\item OS Linux,
	\item Redis server,
	\item Python 3.6+,
	\item SSH server,
	\item Git.
\end{itemize}

Doporučené požadavky:

\begin{itemize}
	\item SQL server (MariaDB, PostgreSQL, \ldots)
	\item WSGI server (uwsgi, gunicorn, \ldots)
	\item Reverzní proxy (nginx, apache, \ldots)
\end{itemize}

\begin{listing}[H]
\caption{Instalace Piper CI core}
\begin{minted}[frame=single]{shell}
# Připravíme systémové požadavky:
...
# Vytvoříme nového systémového uživatele.
# Nutné pro oddělené ~/.ssh/authorized_keys.
useradd piper && su - piper
# Nainstalujeme samotný balíček pomocí pip:
pip install git+https://github.com/francma/piper-ci-core.git
# Nainstalujeme uwsgi server (volitelné):
pip install uwsgi
# Vytvoříme produkční konfiguraci:
cp config.example.yml config.yml
# Upravíme konfiguraci podle instrukcí v souboru:
vim config.yml
# Inicializujeme projekt s ROOT uživatelem:
piper-core config.yml --init "[ROOT_EMAIL] [ROOT_ID_RSA_PATH]"
# Spustíme:
uwsgi --http-socket :[PORT] -w piper_core.run:app --pyargv config.yml
# Obnova databáze uživatelů v authorized_keys.
# Doporučené je jej přidat jako úlohu do cronu.
piper-core --reload config.yml
\end{minted}
\end{listing}

\subsection{Piper CI LXD runner}

Požadavky na systém:

\begin{itemize}
	\item OS Linux,
	\item Python 3.6+,
	\item Git,
	\item LXD.
\end{itemize}

\begin{listing}[H]
\caption{Instalace Piper CI LXD runner}
\begin{minted}[frame=single]{shell}
# Připravíme systémové požadavky:
...
lxd init
# LXD server musí naslouchat na HTTP protokolu:
lxc config set core.https_address [::]
# Vygenerujeme autentizační klíč a certifikát pro LXD:
openssl genrsa 2048 > client.key
openssl req -new -x509 -nodes -sha1 -days 365 -key client.key 
            -out client.crt
# Nainstalujeme samotný balíček pomocí pip:
pip install git+https://github.com/francma/piper-ci-lxd-runner.git
# Vytvoříme produkční konfiguraci:
cp config.example.yml config.yml
# Upravíme konfiguraci podle instrukcí v souboru:
vim config.yml
# Stáhneme obrazy k vytvoření kontejnerů:
lxc image copy images:alpine/3.5 local: --copy-aliases
# Spustíme:
piper-lxd config.yml
\end{minted}
\end{listing}


\subsection{Piper CI web}

Požadavky na systém:

\begin{itemize}
	\item OS Linux,
	\item Python 3.6+.
\end{itemize}

Doporučené požadavky:

\begin{itemize}
	\item WSGI server (uwsgi, gunicorn, \ldots)
	\item Reverzní proxy (nginx, apache, \ldots)
\end{itemize}

\begin{listing}[H]
\caption{Instalace Piper CI web}
\begin{minted}[frame=single]{shell}
# Připravíme systémové požadavky:
...
# Nainstalujeme samotný balíček pomocí pip:
pip install git+https://github.com/francma/piper-ci-web.git
# Vytvoříme produkční konfiguraci:
cp config.example.yml config.yml
# Upravíme konfiguraci podle instrukcí v souboru:
vim config.yml
# Nainstalujeme uwsgi server (volitelné):
pip install uwsgi
# Spustíme
uwsgi --http-socket :[PORT] 
      -w piper_web.run:app
      --pyargv config.yml
\end{minted}
\end{listing}