\chap Návrh architektury systému

Systém CI pod sebou bude spravovat několik Runnerů, každý Runner v~sobě bude spouštět jednotlivé exekutory.
Komunikace bude probíhat přes API.
Jakákoliv funkcionalita by měla být zprvu naimlementovaná jako API endpoint a až pak zavedena do systému.

\sec Části systému

\secc Informační rozhraní

Webové rozhraní. Napsané ve Flasku a nějaké hipster frontend JS knihovně.

\secc Řidič

Zajišťuje komunikaci s~uživatelem, informačním rozhraním a Runnery na pevně daném protokolu.
Řidič by měl být podle vytížení Runnerů (API endpoint na Runneru) a požadovaných služeb rozhodnout, kterému Runneru práci předá.

\secc Runner

Zadaný API vstup rozparsuj pro potřeby Exekutoru a nech ho provést dané příkazy.
Runner by měl spravovat pouze jeden typ exekutoru (UNIX like).
Runner by měl běžet na samostatném stroji, aby neubíral systémové prostředky ostatním.
Runner v~sobě většinou bude spouštět více instancí Exekutoru.

\seccc Exekutor

Konečný vykonavatel příkazů. Řidič by neměl o~existenci exekutorů vědět a komunikovat pouze s~Runnery.
Exekutorem je například VirtualBox, Docker a jiné.
Samotná exekutor by měl být co nejméně modifikovaný.
Nechceme například aby VirtualBox exekutor přijímal pouze image, které mají nainstalovaný nějaký obsáhlý set závislostí.

\secc Cache

Jelikož by cache měla být sdílená mezi Runnery, pak musí fungovat jako samostatná jednotka.
Runner by měl komunikovat přímo s~cache, aby se minimalizovat počet účastníků v~přenosu často velkých souborů.
Prvnotní komunikace bude vázána se systémem (výměna adres).
Je opravdu nutné separovat Cache jako service?

\sec Komunikace systému

Záleží jestli se bude jednat o vnitrosystémovou komunikaci nebo komunikaci systém -- uživatel.
Vnitřní komunikace může používat nějaký \uv{složitější} protokol.

\secc Websocket

Technologie podporující streaming do prohlížečů.

\secc JSON over HTTP

Minimální, fungující v prohlížečích.
Vhodné jako technologie pro public API.
Nevýhodou je nepodpora streamingu (spam requestů je nevyhovující), kterou lze zachránit WebSockety.

\secc GRPC

Pracuje nad JSON, XML, Protobuffers.
Podpora streamingu.
Optimalizace zpráv.
GRPC neprotlačíme do prohlížeče (jedině po překladu na Websocket), vhodné pouze pro vnitrosystémovou komunikaci.

% http://www.grpc.io/docs/tutorials/basic/python.html

\secc GraphQL

\uv{Chytřejší} REST.
Oproti RESTu není orientovaný na zdroje (Uživatelé, články) ale umí pracovat s vazbami (něco jako SQL).
Na jeden dotaz možné získat více různých výsledků.
Facebook.

\begtt
{
 user(id: 1) {
   age
   friends {
     name
   }
 }
}
\endtt
\nobreak \label[graphql-minimal-example]
\caption/l Ukázka GraphQL -- vrať věk uživatele s ID = 1 a jména jeho přátel.
\smallskip

\sec Fronta

Implementace distribuovaných prioritních front ve stylu producenta a konzumenta.
Použití: máme více Runnerů a chceme jim na základě jejich vytížení přiřazovat úkoly.

\secc RabbitMQ

TODO

\secc Redis

TODO

\secc Celery

Používá RabbitMQ, Redis a další jako message broker. 
Primární implementace v Pythonu.

\sec Komuninakce s~běhovým prostředím

Cílem je vytvořit rozhraní s~VM, které přijme zadaný příkaz nebo jejich sadu a rozhraní schopno streamovat výstup z~příkazů.
Rozhraní by mělo udržovat stály stav (proměnné shellu, ...).

\secc Virtualbox

\seccc COM port pro čtení/zápis

Pomocí VirtualBox administračního rozhraní vytvoříme COM port, který nasměrujeme do souboru/socketu.
Každý spuštěný příkaz poté přesměrujeme na zvolený COM port a necháme aplikaci na rodičovském stroji číst výstupy.
Spojení je oboustranné, takže je možné předávat příkazy i dovnitř stroje, které je ale nutné parsovat a předávat systému další aplikací běžící uvnitř VM.
Tento způsob předpokládá už přihlášeného uživatele.

\seccc SSH spojení

SSH je kryptografický protokol, který umožňuje oboustrannou komunikaci mezi účastníky.
Přihlašování uživatelů do VM bez hesla lze docílit uložením jejich veřejných klíčů do souboru {\it known\_hosts}.
Soubor {\it known\_hosts} je uložen v~domovském adresáři uživatele a bude nutno zajistit jeho modifikaci buďto přes dodatečné připojení VDI obrazu disku VM nebo předpřípravou systému.

\seccc Vzdálený terminál přes COM port

Podobné SSH spojení.
Nutná editace konfigurace VM systému, aby daný COM port viděl jako terminálový.
Přihlašování uživatele a spuštění příkazů za pomoci přímého vpisu/čtení ze socketu.

\secc Docker

Příkazy lze předávat do kontenejneru přímo z hostujícího stroje.
Výpis výsledku je přímo na stdout, který lze přesměrovat do roury a z té dál do CI systému.

\begtt
docker start ecstatic_perlman
docker exec -i ecstatic_perlman bash -c 'ls -al' 
\endtt
\nobreak \label[docker-minimal-example]
\caption/l Ukázka Docker exekutoru
\smallskip

\sec Perzistence dat

\secc Souborový systém

Výhodné v~jednoduchosti implementace.
Komplikované prohledávání bez neexistujících indexů.
Limitováno použitým souborovým systémem -- velké množství souborů bude velmi zpomalovat přístupy.

\secc Souborový systém + in-memory cache

Časté přístupy přímo na disk nahradíme přístupy do rychlé in-memory cache.

\secc In-memory + disk dump

Redis dumps nebo log streaming (připis do logu každých x).

\secc Databáze

Optimalizace na úrovni zvolené databáze.
Velké soubory typu logů ukládat na disku a z~databáze na ně pouze odkazovat.

