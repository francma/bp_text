\chapter{Travis CI}

Travis CI je open source služba kontinuální integrace spjatá s verzovací hostovací službou GitHub.

\section{Průběh integrace}

Podobně jako u GitLab CI je konfigurace řešena souborem \verb|.travis.yml|, který je umístěn přímo v~kořenovém adresáři repozitáře projektu.

Proces integrace je rozdělen na fáze (\textit{stages}), úkoly (\textit{jobs}) a příkazy.
Fáze se vykonávají sekvenčně za sebou v~zadaném pořadí.
Selhání fáze znamená zastavení provádění všech následujícíh fází.
Selhání příkazu znamená selhnání celého úkolu, ale ne zastavení provádění tohoto úkolu.
Úkol vždy provede všechny příkazy bez ohledu na to, jestli nějaký předchozí selhal. 
Pokud bychom chtěli, aby se po selhnání příkazu úkol ukončil, pak je nutné zadefinovat proměnnou shellu na začátku skriptů (\verb|set -eo pipefail|).

Každý úkol je spoštěn ve svém vlastním běhovém prostředí.

\begin{listing}[ht]
\begin{minted}[frame=single,linenos]{yaml}
stages:
  - test
  - build
  - deploy
\end{minted}
\caption{Definice fází v .travis.yml}
\end{listing}

Úkol náleží právě do jedné fáze a jeho přiřazení se provede vzoru z ukázky kódu \ref{code:travis-stages}.

\begin{listing}[ht]
\begin{minted}[frame=single,linenos]{yaml}
stages:
  - test
jobs:
  include:
    - stage: test
      script: pytest

\end{minted}
\label{code:travis-stages}
\caption{Definice úkolů s fázemi v .travis.yml}
\end{listing}

Další možností definice úkolů je přes tzv. Build Matrix jak je ukázáno v ukázce kódu \ref{code:travis-matrix}.
Výsledkem této konfigurace bude vytvoření 6 úkolů podle kartézského součinu.
Oba zmíněné přistupy konfigurace lze kombinovat.

\begin{listing}[ht]
\begin{minted}[frame=single,linenos]{yaml}
language: python
python:
  - '3.5'
  - '3.4'
  - '2.7'
env:
  - TOXENV=first
  - TOXENV=second
script:
  - tox
\end{minted}
\label{code:travis-matrix}
\caption{Definice build matrixu v .travis.yml}
\end{listing}

\section{Architektura}

Systém je rozdělen na několk částí, které mezi sebou komunikující přes distribuovanou frontu RabbitMQ.
Datové uložiště je realizováno v SQL databázi PostreSQL.
Pro potřeby cachování je využiván Redis.

\paragraph{Travis Core}

Stará o většinu logiky pro Travis CI.
Obsahuje model aplikace a služby, které jsou sdíleny se zbytkem systému.

\paragraph{Travis API}

Rozhraní překladající HTTP požadavky na volání jádra Travis Core.

\paragraph{Travis Hub}

Shromažďuje údálosti a komunikuje s aplikacemi mimo vnitřní systém.

\paragraph{Travis Listener}

Naslouchá notifikacím z Githubu a informuje o nich zbytek systému přes distribuovanou frontu.

\paragraph{Travis Build}

Překládá uživatelskou konfiguraci integrace na koncový skript, který bude vykonán Travis Workerem.

\paragraph{Travis Worker}

Zodpovědný za spouštění skriptů integrace a tvorbou čistého prostředí pro jejich běh.
Úzce spolupracuje s Travis Loggerem, který ukládá výstup skriptu integrace.

\paragraph{Travis Web}

Obstaravá webové rozhraní pro koncového uživatele.

\paragraph{Travis Tasks}

Stará se o komunikace mezi Travis Hubem a zbytkem vnitřního systému.

\subsection{Běhová prostředí}

Travis CI podporuje několik běhových prostředí vyjmenovaných v následující tabulce \ref{table:travis-env}.
Výhodou kontejnerového prostředí je jeho rychlejší start.

\begin{table}[h]
\centering
\caption{Výčet běhových prostředí v Travis CI}
\label{table:travis-env}
\begin{tabular}{|l|l|l|l|l|}
\hline
& Superuživatel & Technologie & Konfigurace  \\ \hline
Ubuntu Precise & Ano & Virtulní stroj & \verb|sudo: required| \\ 
& & & \verb|dist: precise|  \\ \hline
Ubuntu Trusty & Ano & Virtulní stroj & \verb|sudo: required| \\
& & & \verb|dist: trusty| \\ \hline
Ubuntu Trusty & Ne & Kontejner & \verb|sudo: false|  \\
& & & \verb|dist: trusty| \\ \hline
OS X & Ano & Virtuální stroj & \verb|os: osx|  \\ \hline
\end{tabular}
\end{table}

\begin{listing}[ht]
\begin{minted}[frame=single,linenos]{yaml}
sudo: required
dist: trusty
script:
  - sudo apt add ...
  - pytest
\end{minted}
\caption{Ukázka definice běhového prostředí v .travis.yml}
\end{listing}

\section{Přístup ke zdrojům systému}

% https://docs.travis-ci.com/user/developer/

\section{Postup nasazení}

% https://github.com/travis-ci/travis-core