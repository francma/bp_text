\chapter{Travis CI}

Travis CI je open source služba kontinuální integrace spjatá s verzovací hostovací službou Github.
Travis je naprogramován v Ruby.


\section{Architektura}

Systém je rozdělen na několk částí, které mezi sebou komunikující přes distribuovanou frontu RabbitMQ.
Datové uložiště je realizováno v SQL databázi PostreSQL.
Pro potřeby cachování je využiván Redis.

\paragraph{Travis Core}

Stará o většinu logiky pro Travis CI.
Obsahuje model aplikace a služby, které jsou sdíleny se zbytkem systému.

\paragraph{Travis API}

Rozhraní překladající HTTP požadavky na volání jádra Travis Core.

\paragraph{Travis Hub}

Shromažďuje údálosti a komunikuje s aplikacemi mimo vnitřní systém.

\paragraph{Travis Listener}

Naslouchá notifikacím z Githubu a informuje o nich zbytek systému přes distribuovanou frontu.

\paragraph{Travis Build}

Překládá uživatelskou konfiguraci integrace na koncový skript, který bude vykonán Travis Workerem.

\paragraph{Travis Worker}

Zodpovědný za spouštění skriptů integrace a tvorbou čistého prostředí pro jejich běh.
Úzce spolupracuje s Travis Loggerem, který ukládá výstup skriptu integrace.

\paragraph{Travis Web}

Obstaravá webové rozhraní pro koncového uživatele.

\paragraph{Travis Tasks}

Stará se o komunikace mezi Travis Hubem a zbytkem vnitřního systému.

\section{Průběh integrace}

Podobně jako u Gitlab CI je konfigurace řešena souborem \verb|.travis.yml|, který je umístěn přímo v~kořenovém adresáři repozitáře projektu.

Proces integrace je rozdělen na fáze (\textit{stages}), úkoly (\textit{jobs}) a příkazy.
Fáze se vykonávají sekvenčně za sebou v~zadaném pořadí.
Selhání fáze znamená zastavení provádění všech následujícíh fází.

\begin{minted}[
    frame=single,
    linenos
  ]{yaml}
stages:
  - test
  - build
  - deploy
\end{minted}

Úkol náleží právě do jedné fáze a jeho přiřazení se provede v konfiguraci následovně.

\begin{minted}[
    frame=single,
    linenos
  ]{yaml}
stages:
  - test
jobs:
  include:
    - stage: test
      script: pytest
\end{minted}

Každý úkol je spoštěn ve svém vlastním běhovém prostředí.

Další možností definice úkolů je přes tzv. Build Matrix.

\begin{minted}[
    frame=single,
    linenos
  ]{yaml}
language: python
python:
  - '3.5'
  - '3.4'
  - '2.7'
env:
  - TOXENV=first
  - TOXENV=second
script:
  - tox
\end{minted}

Výsledkem této konfigurace bude vytvoření 6 úkolů podle kartézského součinu.
Oba zmíněné přistupy konfigurace lze kombinovat.


\subsection{Běhová prostředí}

Travis CI podporuje několik běhových prostředí vyjmenovaných v následující tabulce.

\begin{table}[h]
\centering
\caption{My caption}
\label{my-label}
\begin{tabular}{l|llll}
               & Superuživatel & Technologie & Konfigurace  \\ \hline
Ubuntu Precise & Ano & Virtulní stroj & \verb|sudo: required| &  \\ 
& & & \verb|dist: precise| &  \\ \hline
Ubuntu Trusty & Ano & Virtulní stroj & \verb|sudo: required|  &  \\
& & & \verb|dist: trusty| &  \\ \hline
Ubuntu Trusty & Ne & Kontejner & \verb|sudo: false|  &  \\
& & & \verb|dist: trusty| &  \\ \hline
OS X & Ano & Virtuální stroj & \verb|os: osx|  &  \\
\end{tabular}
\end{table}

Výhodou kontejnerového prostředí je jeho rychlejší start.


\begin{listing}[ht]
\begin{minted}[
    frame=single,
    linenos
  ]{yaml}
sudo: required
dist: trusty
script:
  - sudo apt add ...
  - pytest
\end{minted}
\caption{Example from external file}
\label{listing:3}
\end{listing}




% https://github.com/travis-ci/travis-core