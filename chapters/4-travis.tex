\chapter{Travis CI}

Travis CI je služba kontinuální integrace spjatá s Git SCM službou GitHub.

\section{Průběh integrace}

Podobně jako u GitLab CI je konfigurace řešena souborem \verb|.travis.yml|, který je umístěn přímo v~kořenovém adresáři repozitáře projektu.

Proces integrace (\textit{build}) je rozdělen na fáze (\textit{stages}), úkoly (\textit{jobs}) a příkazy.
Fáze se vykonávají sekvenčně za sebou v~zadaném pořadí.
Selhání fáze znamená zastavení provádění všech následujících fází.
Selhání příkazu znamená selhání celého úkolu, ale ne zastavení provádění tohoto úkolu.
Úkol vždy provede všechny příkazy bez ohledu na to, jestli nějaký předchozí selhal. 
Pokud bychom chtěli, aby se po selhání příkazu úkol ukončil, pak je nutné zadefinovat proměnnou shellu na začátku skriptů (\verb|set -eo pipefail|).

Každý úkol je spuštěn ve svém vlastním běhovém prostředí.

\begin{listing}[ht]
\begin{minted}[frame=single,linenos]{yaml}
stages:
  - test
  - build
  - deploy
\end{minted}
\caption{Definice fází v .travis.yml}
\end{listing}

Úkol náleží právě do jedné fáze a jeho přiřazení se provede vzoru z ukázky kódu \ref{code:travis-stages}.
Fáze jsou v Travis CI novým konceptem, který je stále ve fázi veřejného beta testování.
% https://blog.travis-ci.com/2017-05-11-introducing-build-stages

\begin{listing}[ht]
\begin{minted}[frame=single,linenos]{yaml}
stages:
  - test
jobs:
  include:
    - stage: test
      script: pytest

\end{minted}
\label{code:travis-stages}
\caption{Definice úkolů s fázemi v .travis.yml}
\end{listing}

Další možností definice úkolů je přes tzv. build matrix jak je ukázáno v ukázce kódu \ref{code:travis-matrix}.
Výsledkem této konfigurace bude vytvoření 6 úkolů podle kartézského součinu.
Oba zmíněné přístupy konfigurace lze kombinovat.

\begin{listing}[ht]
\begin{minted}[frame=single,linenos]{yaml}
language: python
python:
  - '3.5'
  - '3.4'
  - '2.7'
env:
  - TOXENV=first
  - TOXENV=second
script:
  - tox
\end{minted}
\label{code:travis-matrix}
\caption{Definice build matrixu v .travis.yml}
\end{listing}

\section{Architektura}

Systém je rozdělen na několik částí, které mezi sebou komunikující přes distribuovanou frontu RabbitMQ.
Datové uložiště je realizováno v SQL databázi PostreSQL.
Pro potřeby cachování je využíván Redis.

% https://github.com/travis-ci/travis-core

\paragraph{Travis Core}

Stará o většinu logiky pro Travis CI.
Obsahuje model aplikace a služby, které jsou sdíleny se zbytkem systému.

\paragraph{Travis API}

Rozhraní překladající HTTP požadavky na volání jádra Travis Core.

\paragraph{Travis Hub}

Shromažďuje údálosti a komunikuje s aplikacemi mimo vnitřní systém.

\paragraph{Travis Listener}

Naslouchá notifikacím z Githubu a informuje o nich zbytek systému přes distribuovanou frontu.

\paragraph{Travis Build}

Překládá uživatelskou konfiguraci integrace na koncový skript, který bude vykonán Travis Workerem.

\paragraph{Travis Worker}

Zodpovědný za spouštění skriptů integrace a tvorbou čistého prostředí pro jejich běh.
Úzce spolupracuje s Travis Loggerem, který ukládá výstup skriptu integrace.

\paragraph{Travis Web}

Obstaravá webové rozhraní pro koncového uživatele.

\paragraph{Travis Tasks}

Stará se o komunikace mezi Travis Hubem a zbytkem vnitřního systému.

\subsection{Běhová prostředí}

Travis CI podporuje několik běhových prostředí vyjmenovaných v následující tabulce \ref{table:travis-env}.
Výhodou kontejnerového prostředí je jeho rychlejší start.

\begin{table}[ht]
\centering
\caption{Výčet běhových prostředí v Travis CI}
\label{table:travis-env}
\begin{tabular}{|l|l|l|l|l|}
\hline
& Superuživatel & Technologie & Konfigurace  \\ \hline
Ubuntu Precise & Ano & Virtuální stroj & \verb|sudo: required| \\ 
& & & \verb|dist: precise|  \\ \hline
Ubuntu Trusty & Ano & Virtuální stroj & \verb|sudo: required| \\
& & & \verb|dist: trusty| \\ \hline
Ubuntu Trusty & Ne & Kontejner & \verb|sudo: false|  \\
& & & \verb|dist: trusty| \\ \hline
OS X & Ano & Virtuální stroj & \verb|os: osx|  \\ \hline
\end{tabular}
\end{table}

Běhové prostředí je pak přizpůsobeno pro konkrétní projekt podle programovacího jazyka konfiguračním klíčem \verb|language|.
Upřesnění verze se pak docílí definicí build matrixu \ref{code:travis-matrix}.
Problém nastává pokud Travis nepodporuje námi vyžadovaný jazyk, který není podporován verzí Ubuntu na Travisu.
Příkladem je \TeX{}live balíček pouze ve verzi 2013 proti aktuální 2017.

\begin{listing}[ht]
\begin{minted}[frame=single,linenos]{yaml}
sudo: required
dist: trusty
language: python
script:
  - sudo apt add ...
  - pytest
\end{minted}
\caption{Ukázka definice běhového prostředí v .travis.yml}
\end{listing}

\section{Přístup ke zdrojům systému}

Travis CI podobně jako GitLab umožňuje přístup ke zdrojům systému pomocí HTTP API.

\section{HTTP API}

% https://developer.travis-ci.org/
Travis API nám dává možnost přistupovat ke zdrojům systému skrze HTTP API.
K autentizaci se používá tokenu, který je součástí HTTP hlaviček.
K reprezentaci dat API se používá JSON formátu.

\begin{table}[ht]
\centering
\fontsize{9.5}{11.5}\selectfont
\caption{Travis API}
\label{table:gitlab-api}
\begin{tabular}{|l|l|l|}
\hline
HTTP metoda + akce                                          & Popis \\ \hline
\verb|GET  /repo/[id]/builds|                               & Vrátí seznam integrací repozitáře      \\ \hline
\verb|GET  /build/[id]/jobs|                                & Vrátí seznam úkolů integrace      \\ \hline
\verb|POST /job/[id]/cancel|                                & Zruší probíhající úkol      \\ \hline
\verb|POST /job/[id]/restart|                               & Restartuje úkol      \\ \hline
\verb|POST /repo/[id]/[branch]/requests|                    & Vytvoří novou integraci      \\ \hline
\verb|GET  /build/[id]|                                     & Vrátí integraci      \\ \hline
\verb|POST /build/[id]/cancel|                              & Zruší probíhající integraci      \\ \hline
\verb|POST /build/[id]/restart|                             & Restartuje probíhající integraci      \\ \hline
\verb|GET  /stage/[id]|                                     & Vrátí fázi      \\ \hline
\verb|GET  /build/[id]/stages|                              & Vrátí seznam fází integrace      \\ \hline
\end{tabular}
\end{table}

\section{Terminálové rozhraní}

% http://www.rubydoc.info/gems/travis/1.8.8
Terminálová aplikace Travis CI, která využívá HTTP API.
Autentizace je prováděna pomocí tokenu nebo uživatelského jména a hesla.
Aplikace má několik zajímavých vlastností, které jí činí uživatelsky přívětivou.
Například pokud jí pustíme ze složky, ve které je se nachází Git repozitář projektu spravovaném Travisem, tak vykonané příkazy se budou vázat právě k tomuto repozitáři.
Další zajímavou vlastností je příkaz \verb|travis whatsup|, který zobrazí poslední integrace všech projektů.
Terminálové rozhraní také dovoluje sledovat log výstup integrací v reálném čase. 

\section{Webové rozhraní}

Travis disponuje také webovým rozhraním, ve kterém lze sledovat průběh integrací.
Data ve webovém rozhraní se automaticky obnovují bez nutného zásahu uživatele.
Implementačně je toho docíleno websocketovým spojením.
Stejně jako terminálové rozhraní, tak i to webové dokáže sledovat log výstupy integrací v reálném čase.

\section{Postup nasazení}

Ač je Travis CI otevřený software, tak provoz na vlastní infrastruktuře je spíše nedoporučován.
