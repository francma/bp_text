\section{Testování softwaru}

Téměř každý programátor při psaní programů píše také testy.
Příkladem takového testu je i jednoduché vypsání výstupu funkce a následné ověření výsledku (ukázka kódu \ref{test-basic-code}).

\begin{listing}[ht]
\caption{\label{test-basic-code}Primitivní test}
\begin{minted}[frame=single,linenos]{python}
def mult(a, b):
    return a * b

# Opravdu mi toto vrátí 6?
print(mult(3, 2))
\end{minted}
\end{listing}

Po vyzkoušení několika vstupů a ověření korektního fungování test smaže a začne se věnovat jiné části programu.
V budoucnu se rozhodne tento kód revidovat (ukázka kódu \ref{test-optimized-code}) a znovu napíše testy, čímž dojde k plýtvání času.

\begin{listing}[ht]
\caption{\label{test-optimized-code}Optimalizace kódu}
\begin{minted}[frame=single,linenos]{python}
def mult(a, b):
    if a == 0 or b == 0:
        return 0

    return a * b

# Bude výsledek opět 6?
print(mult(3, 2))
\end{minted}
\end{listing}

Lepším přístupem je testy zachovávat a při vývoji je používat k průběžné kontrole (ukázka kódu \ref{test-test}).
Pokrytím kódu testy dojde k zefektivnění vývoje a umožní odhalit chyby včas než budou zavedené do zbytku systému.

\begin{listing}[ht]
\caption{\label{test-test}Ukázka napsaného testu}
\begin{minted}[frame=single,linenos]{python}
from project import mult

assert mult(3, 2) == 6
\end{minted}
\end{listing}

K ulehčení psaní testů můžeme využít některé z knihoven.

\subsection{Typy testů}

Testy dělíme na dva typy -- integrační a jednotkové.

Úkolem jednotkového testu je otestovat právě jednu komponentu (třídu, funkci, modul, \ldots) v izolaci od ostatních komponent.
Tohoto docílíme tak, že závislé komponenty nahradíme komponentami \uv{falešnými}, které simulují určité deterministické chovaní.
Například se jedná o simulaci odeslání HTTP požadavku.
Místo vytváření spojení a čekání na odpověď můžeme nahradit zodpovědnou komponentu tak, že vždy vrátí kladný návratový HTTP kód s určitou odpovědí.

Testováním více komponent, bez pomoci komponent \uv{falešných}, vytvoříme test integrační.
V našem uvedeném případě bychom použili reálný HTTP server.

Výhodou jednotkových testů oproti integračním je jejich vyšší rychlost (například ušetříme čas potřebný na navázání spojení k serveru).
Další výhodou je i jednodušší simulace mezních stavů (například server vrátí konkrétní chybovou hlášku).
Jednotkové testy nám také dovolují testovat komponenty v případě, že jejich závislosti v podobě komponent nejsou ještě dokončené.
Naopak test integrační lépe testuje reálné prostředí.

% https://cs-blog.petrzemek.net/2015-04-18-proc-rozlisovat-jednotkove-a-integracni-testy