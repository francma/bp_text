\section{Testování softwaru}

Téměř každý programátor při psaní programů současně píše také testy.
Příkladem takového testu je i jednoduché vypsání návratové hodnoty funkce a následné ověření výsledku (ukázka kódu \ref{test-basic-code}).
Po vyzkoušení několika vstupů a ověření korektního fungování je tento test programátorem smazán.
Nicméně je velmi pravděpodobné, že v budoucnu se další člen projektu rozhodne tento kód revidovat a pro ověření jeho funkčnosti opět napíše testy.

\begin{listing}[ht]
\caption{\label{test-basic-code}Primitivní test}
\begin{minted}[frame=single,linenos]{python}
def mult(a, b):
    return a * b

# Opravdu mi toto vrátí 6?
print(mult(3, 2))
\end{minted}
\end{listing}

Opětovným psaním testů je zbytečně ztrácen čas, a proto je lepší tyto testy zachovávat a při vývoji je používat k průběžné kontrole (viz. ukázka kódu \ref{test-test}).
Pokrytím kódu testy dojde k zefektivnění vývoje a možnosti odhalit chyby než budou zavedené do zbytku systému.

\begin{listing}[ht]
\caption{\label{test-test}Ukázka napsaného testu}
\begin{minted}[frame=single,linenos]{python}
from project import mult

assert mult(3, 2) == 6
assert mult(3, 3) == 9
assert mult(0, 3) == 0
\end{minted}
\end{listing}

\subsection{Rozdělení testů dle typu}

Testů existuje několik typů.
Nejprve se zaměříme na testy integrační a jednotkové.
Úkolem jednotkového testu je otestovat právě jednu komponentu (třídu, funkci, modul, \ldots) v izolaci od ostatních komponent.
Izolace docílíme tak, že závislé komponenty nahradíme komponentami \uv{falešnými} (tzv. \uv{mocks}, \uv{stubs} \cite{fowler_mocks}), které se chovají \uv{deterministicky} dle našeho zadání (ukázka \ref{code:test-stub}).
Použití \uv{falešných} komponent nám umožňuje snadno simulovat mezní stavy (například chyba připojení k databázi), nebo mohou sloužit jako dočasná náhrada za dosud neimplementované části systému.
Dalším přínosem je urychlení vykonání testů (není nutné se připojovat k databázi).
\cite{zemek_integracni_x_jednotkove}

\begin{listing}[ht]
\caption{\label{code:test-stub}Ukázka použití falešné třídy}
\begin{minted}[frame=single,linenos]{python}
class Database:
    def get(self, idx):
        return self.db.query('SELECT * FROM abc WHERE id = %i' % idx)

# Falešná třída pro testovací učely
class FakeDatabase:
    def get(self, idx):
        return {'id': idx, 'property': 'test' + str(idx)}

class Foo:
    def __init__(self, database):
        self.database

    def bar(self, idx):
        return database.get(idx)['property']

foo = Foo(FakeDatabase)
assert foo.bar(1) == 'test1'
assert foo.bar(2) == 'test2'
\end{minted}
\end{listing}

Doplňující jednotkových testů jsou testy integrační, jejichž úkolem je otestovat komponenty včetně jejich provázání.
Integračními testy se blížíme k reálnému produkčnímu použití, tudíž by neměly být opomíjeny.

Testování ale nemusí podléhat pouze samotná funkčnost programu.
Příkladem může být statická analýza programu, výkonnostní, akceptační testy a další.
Častým předmětem testování je i kontrola stylu zápisu programu proti sadě pravidel (tzv. \uv{coding style})

