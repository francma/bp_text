\begin{introduction}

Kontinuální integrace (CI) je technika vývoje softwaru, která vývojářům pomáhá odhalit chyby v jejich programech.
Technika spočívá v časté integraci (slučování) kódu ve společném repozitáři projektu, který je následně podroben integračním a jednotkovým testům.
Úkolem kontinuální integrace není definice samotných testů (to je stále úkolem vývojářů projektu), ale jejich vynucené spuštění při každé integraci.
Vývojář nemusí mít vždy možnosti spustit testy na svém lokálním prostředí (například testování pro více platforem) nebo se může jednat o prosté opomenutí nebo dokonce přehranou důvěru ve svůj vyprodukovaný kód.
Systém kontinuální integrace pak funguje jako hráz, která zabraňuje zanesení chyb do produkčního prostředí.
Kontinuální integrace také dopomáhá k časnému odhalení chyb a tím pádem snižuje náklady na jejich opravení.

Samotný systém kontinuální integrace, lze definovat jako množinu softwaru realizující následující úkony:

\begin{enumerate}
	\item Naslouchat a reagovat na změny v~repozitáři projektu.
	\item Připravit běhové prostředí vhodné pro testovací proces.
	\item Stáhnout zdrojové soubory projektu z~repozitáře.
	\item Sestavit stažený projekt včetně jeho závislostí.
	\item Spustit testovaní dle zadání projektu.
	\item Informovat vývojáře a další systémy o~výsledku testování.
\end{enumerate}

Cílem této práce je navrhnout minimalistický systém kontinuální integrace, který bude určen pro menší komunitní projekty.
Hlavním rysem tohoto systému bude jeho minimálnost a jednoduchost, čímž docílíme snadného rozšíření tohoto systému o další funkcionalitu a připravíme půdu pro integraci s dalšími systémy.

V této práci se nejprve budeme zabývat problematikou úzce spjatou s kontinuální integrací a to testováním softwaru, verzovacími nástroji a běhovými prostředími.
V následující části analyzujeme existující open-source systémy kontinuální integrace a následně navrhneme systém vlastní.
Námi navržený systém následně naimplementujeme a otestujeme.

\end{introduction}
