\begin{introduction}

Programování a procesy s~ním spojené prochází nepřetržitým vývojem, který tuto disciplínu stále mění. 
Přidáváním vyšších vrstev abstrakce se vzdalujeme od stroje samotného a přecházíme k~simulaci fungování reálného světa.
Vývoj softwaru je také podporován různými nástroji a technikami, které si kladou za cíl snížení nákladů.
Jednou z takových technik je i technika kontinuální integrace.

Kontinuální integrace (CI) je technika vývoje softwaru, která spočívá v časté integraci (slučování) práce mezi členy projektu ve společném repozitáři.
Součástí každé integrace je sestavení projektu a následné spuštění testů dle zadání.
Časté integrace spolu s kvalitně napsanými testy vedou k brzkému odhalení chyb a tím pádem ke snížení nákladů na jejich opravení.
\cite{fowler_ci}

\section{Cíle a obsah práce}

Cílem této práce je navrhnout a implementovat minimalistický systém CI vhodný pro menší a komunitní projekty.
Při návrhu systému budeme dbát na jeho jednoduchost, modularitu a možnou integraci s dalšími systémy.

Tato práce nejprve uvede do problematiky testování, verzovacích nástrojů a běhových prostředí jakožto základních stavebních kamenů kontinuální integrace.
Práce se následně bude zabývat analýzou existujících řešení -- konkrétně Gitlab~CI, Travis~CI a Buildbot.
Práce následně provede čtenáře návrhem a popisem použitých technologií, implementací a následným testováním vlastního CI systému.

\end{introduction}
