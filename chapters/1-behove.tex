\chapter{Běhová prostředí}

Běhovým prostředím je myšleno prostředí, ve kterém je prováděny integrační testy.
Toto prostředí by mělo být:

\begin{itemize}
	\item Uvedeno do počátečního stavu na začátku každé integrace.
	\item Odolné vůči vnějším vlivům prostředí.
	\item Izolované proti neoprávněným zásahům do vnějšího systému.
\end{itemize}

Těchto požadavků lze dosáhnout vytvořením virtualizovaním prostředí, které nám zajistí izolaci integrace.
Virtualizace je technika při které se jeden fyzický prostředek transformuje na několik virtuálních prostředků.
Fyzický prostředek slouží jako hostitel pro virtuální prostředky.

Virtualizační prostředí uvedeme do počátečního stavu jednoduše jeho znovuvytvořením.

% https://edux.fit.cvut.cz/oppa/BI-ADU/prednasky/BIADU_P11_Virt.pdf

\section{Plná virtualizace}

% Obrazek hypervizor / os / virtualni stroje

Výsledkem plné virtualizace je virtuální stroj, který simuluje hardware, na kterém je možné provozovat další instanci operačního systému.
Prostředníkem mezi virtuálním strojem a hostitelským systémem se nazývá hypervizor.

\imagefigurelarge{hypervizor.pdf}{Hypervizor typu 1 (vpravo) proti typu 2 (vlevo)}

Virtualizační programy: VirtualBox, QEMU

\subsection{Softwarová virtualizace}

Softwarová virtualizace funguje na principu úpravy instrukcí virtualizovaného stroje.
Je nutné přepsat privilegované instrukce na neprivilegované a upravit přístupy do paměti.

\subsection{Hardwarové asistovaná virtualizace}

Hardwarové asistovaná virtualizace funguje na principu simulace hardwaru za podpory specializovaných instrukcí procesoru.
Na tomto simulovaném hardwaru je pak možné spustit další instanci operačního systému, která bude kompletně odstíněna od svého hostitele.
Nevýhodou tohoto řešení je nutnost alokace systémových prostředků na běh dalšího operačního systému i samotnou simulaci hardwaru.

\section{Virtualizace na úrovni operačního systému}

% http://www.linfo.org/kernel_space.html
Systémová paměť moderních operačních systémů je dělena na dvě části: prostor jádra a uživatelský prostor.
Prostor jádra je přímo využíván pro běh jádra operačního systému a jeho služeb.
V~uživatelském prostoru běží procesy spuštěné uživatelem bez přístupu do prostoru jádra.
Uživatelské procesy komunikují s~prostorem jádra pouze skrz systémové metody.
Na x86 procesorech se k~rozdělení paměti používá chráněný režim, ve kterém lze stránky virtuální paměti rozdělit podle privilegovanosti.

Virtualizace na úrovni operačního systému je virtualizační technika, která dovoluje koexistenci několika oddělených uživatelských režimů, které fungují nad stejným prostorem jádra.
Izolovanému uživatelskému prostoru se často označuje jako tzv. \uv{kontejner}.

Výhodou je menší režie oproti plné virtualizaci, protože už není nutné spouštět celou novou instanci operačního systému.
Oproti plné virtualizaci je naopak nutné zajistit, aby se jednotlivé uživatelské režimy nemohly neoprávněně ovlivňovat.
Toto je netriviální úloha a proto je vhodné zavést některá bezpečností opatření.
Jedním z~opatření je vytvoření kontejneru pod neprivilegovaným uživatelem (ne-root, ne-administrátor), takže případní \uv{únik} z~kontejneru nezpůsobí velkou nebo žádnou škodu.
Dalším opatřením může být spuštění nové instance operačního systému pomocí plné virtualizace, která bude využívána pro běh kontejnerů.
Únikem z~kontejneru se pak útočník dostane pouze do virtualizovaného operačního systému bez možnosti ovlivnit hostitele.

{
Mezi technologie pro tvorbu kontejnerů patří:

\begin{itemize}
	\item Jmenné prostory a kontrolní skupiny (Linux)
	\item Jails (BSD)
	\item Hyper-V
\end{itemize}
}

\subsection{Nástroje operačního systému Linux}

Linux kernel sám o sobě nenabízí přímé kontejnerové řešení, ale pouze sadu nástrojů, pomocí kterých je kontejner vytvořen.
Jedná se o jmenné prostory a kontrolní skupiny.
Nad těmito nástroji je postaveno například kontejnerové řešení LXC nebo Docker.

\subsubsection{Jmenné prostory (namespaces)}

Nástroj Linux kernelu sloužící k~vytvoření izolovaných skupin procesů.
Každý proces má svůj jmenný prostor, který mu vymezuje viditelnost pouze na vlastní prostředky a prostředky podřízených jmenných prostorů.
Při startu systému existuje nejméně jeden namespace.
Jmenné prostory tvoří hierarchickou stromovou strukturu.
Existuje několik druhů jmenných prostorů.

\paragraph{PID}

Proces vidí procesy pouze ze svého a podřízených PID jmenných prostorů.

\paragraph{NET}

Síťové rozhraní náleží právě do jednoho jmenného prostoru a není sdíleno s~podřízenými jmennými prostory.
Každé síťové rozhraní má svoje IP adresy, routovací tabulky, firewall a další síťové prostředky. 

\paragraph{MNT}

Kontrolujeme jaké mount pointy budou viditelné.
Vytvořením nového MNT namespacu přeneseme všechny mount pointy rodiče.
Změny v~aktuálním namespacu neovlivňují rodičovský namespace. 


\paragraph{IPC}

Meziprocesová komunikace.
Linux kernel ve výchozím stavu přiděluje sdílenou paměť na základě uživatelského ID, což by vedlo ke kolizím mezi jmennými prostory.

% FIXME? PostreSQL pokud má stejné UID, pak přistupuje do stejné paměti a koliduje??

\paragraph{UTS}

Dovoluje systému vystupovat pod více hostitelskými jmény (hostnames).

\paragraph{user}

Umožňuje libovolně mapovat uživatelská ID z~kontejneru na ID uživatelů z~hostitele.
Uživatel v~kontejneru se pak může tvářit jako root i když vně kontejneru se jedná pouze o~obyčejného uživatele (bez práv roota).

% https://www.youtube.com/watch?v=sK5i-N34im8

O~vytvoření jmenného prostoru se stará systémové volání FORK.

\subsubsection{Kontrolní skupiny (cgroups)}

Nástroj Linux kernelu umožňující přidělovat skupinám procesů limit na systémové prostředky a prioritizovat jednotlivé skupiny při přístupu k~nim.
Typy prostředků jsou:
\begin{itemize}
	\item CPU čas
	\item Operační paměť
	\item Disková paměť % FIXME tohle je divny nazev
	\item Síťový provoz
	\item Provoz na disku
\end{itemize}

Mezi další vlastnosti kontrolních skupin patří možnost zmrazit skupinu procesů v~určitém stavu a následně je znovu odmrazit. % FIXME mam tohle vubec rozebirat


\subsubsection{Copy on write}

Technika optimalizace správy dat, při které se při kopírování dat namísto vytvoření jejich kopie pouze označí jako sdílená.
Fyzickou kopii dat je třeba vytvořit až v~okamžiku jejich modifikace.
Za cenu vyšší režie můžeme takto ušetřit mnoho paměti.

% https://books.google.cz/books?id=9yIEji1UheIC&lpg=PA295&dq="copy+on+write"&pg=PA295&redir_esc=y#v=onepage&q="copy%20on%20write"&f=false

Použití v~Linuxu v systémovém volání FORK nebo v~různých filesystémech (ZFS).

\subsection{BSD jails}

Kontejnerová technologie nacházející v operačních systémech typu BSD (FreeBSD, DragonflyBSD, \ldots).
Na rozdíl od Linuxových kontejnerů, kde jsme si museli poskládat konečný kontejner z jednotlivých nástrojů, nám BSD nabízí přímo sadu systémových volání (\verb|jail_*|).

Kontejner nám zajištujě izolaci:
\begin{itemize}
	\item Proces vidí procesy ze svého kontejneru
	\item Modifikace síťového nastavení zevnitř kontejneru není povolena (pevná IP adresa, routovací tabulky)
	\item Kontejner nemá přímý přístup k síťovému socketu (ICMP protokol nebude fungovat)
	\item Kontejner je vymezen pouze na adresářovou strukturu, ve které byl založen
\end{itemize}

Kontejneru lze také omezovat systémové prostředky (RAM, disk, síť, \ldots)

% https://www.freebsd.org/doc/handbook/jails.html


\subsection{Hyper-V}

Proprietární technologie firmy Microsoft.
Dostupné pouze na některých verzích Windows (profesionální verze, dražší oproti klasickým verzím).
Kontejnerové řešení Docker je schopné využít tuto technologii a fungovat nad ní jako frontend.
