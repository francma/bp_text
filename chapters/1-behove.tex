\chapter{Běhová prostředí}

Proces integrace je potřeba spustit v běhovém prostředí, které jsme pro tento účel připravili.
Z bezpečnostních hledisek by mělo být toto běhové prostředí:

\begin{itemize}
	\item Odolné vůči vnějším vlivům prostředí.
	\item Izolované proti neoprávněným zásahům do vnějšího systému.
	\item Limitované na zdroje.
\end{itemize}

Nechceme například, aby byl záškodník schopen za pomoci integrace kompromitovat nesouvisející části systému nebo použíl proces integrace pro útok na externí systémy.

Dalším požadavkem je, že integrace se stejnými vstupními daty by měla vždy vracet stejné výsledky.
Ač se zdá toto pravidlo jasné, tak jeho zaručení nemusí být úplně přímočaré.
Pokud například všechny integrace provádíme ve stejném běhovém prostředí a během jedné integrace se poškodí systém, tak všechny další integrace budou selhávat.

Těchto požadavků lze dosáhnout vytvořením virtualizovaného prostředí, které nám zajistí izolaci integrace.
Virtualizace je metodika dělení fyzických prostředků počítače do více virtuálních běhových prostředí, aplikací jednoho nebo více konceptů a technologií, jako je hardwarové a softwarové izolace, částečná nebo úplná simulace hardwaru, emulace, QoS a dalších \cite{virt_intro}.

\section{Plná virtualizace}

Výsledkem plné virtualizace je virtuální stroj, který simuluje hardware, na kterém je možné provozovat další instanci operačního systému.
Takto provozovaná instance OS se nazývá host a systém, pod kterým host běží, hostitel.

Prostředníkem mezi hostem a hostitelem je hypervizor.
Hypervizor je software nebo firmware, který dovoluje virtualizovat systémové prostředky.
Existují 2 typy hypervizorů.
Hypervizor typu 1 běží přímo nad hardwarem zatímco hypervizor typu 2 běží nad hostitelským OS, který řeší služby potřebné k virtualizaci samotné jako například přístup ke vstupu/výstupu, správa paměti a další.
\cite{hypervisor_def}

Nevýhodou plné virtualizace nemožnost měnit množství přidělených prostředků (velikost RAM, množství CPU) virtuálnímu stroji za jeho běhu.

Mezi hypervizory například patří VMware, KVM a z/VM.

\subsection{Softwarová virtualizace}

Softwarová virtualizace spoléhá na úpravu strojových instrukcí za běhu hostovaného stroje.
Jedná se například o překlad privilegovaných instrukcí na neprivilegované a úpravy přístupů do paměti.
Z těchto důvodů je tento typ plné virtualizace velmi pomalý.

\subsection{Hardwarové asistovaná virtualizace}

Jedná se o virtualizaci, která je podpořena procesorem hostitelského systému.
Použití hardwarově asistované virtualizace přináší oproti softwarové virtualizaci velký skok ve výkonu.
Na CPU architektuře x86 se jedná o technologie VT-x (Intel) \cite{vt_x} a AMD-V (AMD) \cite{amd_v}.
Princip spočívá v rozšíření instrukční sady CPU o další instrukce, které dovolují procesoru přepínat mezi více kontexty.
Tento přístup dovoluje provozovat nemodifikované instance OS uvnitř virtuálního prostředí.

\section{Paravirtualizace}

Paravirtualizace je technika, při které je hostovaný systém upraven tak, aby místo namísto přímé komunikace s hardwarem používal API hostitele.
Výhodou tohoto přístupu byla snížená režie oproti plné softwarové virtualizaci.
S příchodem hardwarové virtualizace, která dokáže být stejně rychlá jako paravirtualiazace se od této technologie postupně upouští \cite{paravirtualization_leave}.

% http://blogs.vmware.com/guestosguide/2009/09/vmi-retirement.html

\section{Virtualizace na úrovni operačního systému}

Systémová paměť moderních operačních systémů je dělena na dvě části: prostor jádra a uživatelský prostor.
Prostor jádra je přímo využíván pro běh jádra operačního systému a jeho služeb.
V~uživatelském prostoru běží procesy spuštěné uživatelem bez přístupu do prostoru jádra.
Uživatelské procesy komunikují s~prostorem jádra pouze skrze systémové metody.
\cite{kernel_space}

% Na x86 procesorech se k~rozdělení paměti používá chráněný režim, ve kterém lze stránky virtuální paměti rozdělit podle privilegovanosti.

Virtualizace na úrovni operačního systému je virtualizační technika, která dovoluje koexistenci několika oddělených uživatelských prostorů, které fungují nad stejným prostorem jádra.
Izolovaný uživatelský prostor se často označuje jako tzv. \uv{kontejner}.

Výhodou je menší režie oproti plné virtualizaci, protože už není nutné spouštět celou novou instanci operačního systému.
Oproti plné virtualizaci je naopak nutné zajistit, aby se jednotlivé uživatelské režimy nemohly neoprávněně ovlivňovat.
Toto je netriviální úloha a proto je vhodné zavést některá bezpečností opatření.
Jedním z~opatření je vytvoření kontejneru pod neprivilegovaným uživatelem (ne-root, ne-administrátor), takže případní \uv{únik} z~kontejneru nezpůsobí škodu.
Dalším opatřením může být spuštění nové instance operačního systému pomocí plné virtualizace, která bude využívána pro běh kontejnerů.
Únikem z~kontejneru se pak útočník dostane pouze do virtualizovaného operačního systému bez možnosti ovlivnit původního hostitele.

Jelikož kontejner sdílí stejný prostor jádra se svým hostitelem, tak není možné, aby byl kontejner založen na jiném OS než hostitel.
Není tedy například možné provozovat Linuxové kontejnery na systému s OS Windows.
Mezi technologie pro tvorbu kontejnerů patří:

\begin{itemize}
	\item jmenné prostory a kontrolní skupiny (Linux),
	\item jails (BSD).
\end{itemize}

\subsection{Nástroje operačního systému Linux}

Linux kernel sám o sobě nenabízí přímé kontejnerové řešení, ale pouze sadu nástrojů, pomocí kterých je kontejner vytvořen.
Jedná se o jmenné prostory a kontrolní skupiny.
Nad těmito nástroji je postaveno například kontejnerové řešení LXC nebo Docker.

Zjednodušeně se jedná o založení nového rootu adresářové struktury (Linux příkaz \verb|chroot|), následným spuštěním kontejneru pracující nad touto adresářovou strukturou a kontrolou přístupu ke zdrojům hostitelského systému.
Vzory této adresářové struktury se nazývají obrazy (\textit{images}).

\subsubsection{Jmenné prostory (namespaces)}

Nástroj Linux kernelu sloužící k~vytvoření izolovaných skupin procesů.
Každý proces má svůj jmenný prostor, který mu vymezuje viditelnost pouze na vlastní prostředky a prostředky podřízených jmenných prostorů.
Jmenné prostory tvoří hierarchickou stromovou strukturu.
Při startu systému existuje nejméně jeden jmenný prostor.
Existuje několik druhů jmenných prostorů.

\paragraph{PID}

Proces vidí procesy pouze ze svého a podřízených PID jmenných prostorů.
\cite{pid_namespaces}

\paragraph{NET}

Síťové rozhraní náleží právě do jednoho jmenného prostoru a není sdíleno s~podřízenými jmennými prostory.
Každé síťové rozhraní má svoje IP adresy, routovací tabulky, firewall a další síťové prostředky. 
\cite{namespaces}

\paragraph{MNT}

Kontrolujeme jaká připojená zařízení (\textit{mount point}) budou viditelné.
Vytvořením nového MNT jmenného prostoru přeneseme všechna připojená zařízení rodiče.
Změny v~aktuálním jmenném prostory neovlivňují jeho rodiče.
\cite{mnt_namespaces}

\paragraph{IPC}

Komunikace mezi procesy.
Linux kernel ve výchozím stavu přiděluje sdílenou paměť na základě uživatelského ID, což by vedlo ke kolizím mezi jmennými prostory.
\cite{namespaces}

\paragraph{UTS}

Dovoluje systému vystupovat pod více hostitelskými jmény (\textit{hostnames)}.
\cite{namespaces}

\paragraph{user}

Umožňuje libovolně mapovat uživatelská ID z~kontejneru na ID uživatelů z~hostitele.
Uživatel v~kontejneru se pak může tvářit jako root i když vně kontejneru se jedná pouze o~obyčejného uživatele (bez root práv).
\cite{user_namespaces}

% https://www.youtube.com/watch?v=sK5i-N34im8

O~vytvoření jmenného prostoru se stará systémové volání \verb|fork|.

\subsubsection{Kontrolní skupiny (cgroups)}

Nástroj Linux kernelu umožňující přidělovat skupinám procesů limit na systémové prostředky a prioritizovat jednotlivé skupiny při přístupu k~nim.
Mezi typy prostředků patří:

\begin{itemize}
	\item CPU čas,
	\item operační paměť,
	\item disková paměť,
	\item síťový provoz,
	\item provoz na disku.
\end{itemize}

\subsubsection{Copy on write}

Technika optimalizace správy dat, při které se při kopírování dat namísto vytvoření jejich kopie pouze označí jako sdílená.
Fyzickou kopii dat je třeba vytvořit až v~okamžiku jejich modifikace.
Za cenu vyšší režie můžeme takto ušetřit mnoho paměti.
\cite{copy_on_write}

Použití v~Linuxu v systémovém volání \verb|fork| \cite{fork_manual} nebo v~různých filesystémech (ZFS, BTRFS) \cite{fs_cow}.

\subsection{BSD jails}

Kontejnerová technologie nacházející v operačních systémech typu BSD (FreeBSD, DragonflyBSD, \ldots).
Na rozdíl od Linuxových kontejnerů, kde se konečný kontejner poskládal z jednotlivých nástrojů, nám BSD nabízí přímo sadu systémových volání (\verb|jail_*|).
Kontejner nám zajišťuje izolaci:

\begin{itemize}
	\item Proces vidí procesy ze svého kontejneru.
	\item Modifikace síťového nastavení zevnitř kontejneru není povolena (pevná IP adresa, routovací tabulky).
	\item Kontejner nemá přímý přístup k síťovému socketu (ICMP protokol nebude fungovat).
	\item Kontejner je vymezen pouze na adresářovou strukturu, ve které byl založen.
\end{itemize}

Kontejneru lze také omezovat systémové prostředky (RAM, disk, síť, \ldots)

% https://www.freebsd.org/doc/handbook/jails.html

