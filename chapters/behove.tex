\section{Běhová prostředí}

Při opakovaném spouštění testů v rámci různých integrací se můžeme dostat do stavu, kdy se integrace budou mezi sebou nechtěně ovlivňovat.
Například nedůkladné vyčištění dočasných souborů pak může vést k falešně negativnímu nebo falešně pozitivnímu výsledku testování.

Další problém může nastat v případě, že repozitář projektu je veřejně dostupný a dovoluje otevření tzv. \textit{pull requestu}.
Útočník toho může využít k modifikaci testovacího procesu s cílem poškodit systém.
Mimo útoku na náš systém může útočník využít systémových prostředků například k útoku na další systémy (DDoS útok).
Také se může stát, že k poškození systému dojde neúmyslně v důsledku fatální chyby při provádění testů.

Uvedené problémy můžeme vyřešit vytvořením virtualizovaného běhového prostředí.
Virtualizace je metodika dělení fyzických prostředků počítače do více virtuálních běhových prostředí, pomocí jednoho nebo více konceptů a technologií, jako je hardwarové a softwarové izolace, částečná nebo úplná simulace hardwaru, limitace prostředků a dalších \cite{virtualization_intro}.
Problém s čištěním dočasných souborů, vytvořených během integrace vyřešíme tak, že každé integraci jednoduše vytvoříme prostředí nové.

\subsection{Plná virtualizace}

Plná virtualizace vytváří virtuální stroj, který simuluje hardware, na kterém je možné provozovat další instanci operačního systému.
Virtualizovaná instance OS se nazývá host (\textit{guest}) a systém, pod kterým host běží, hostitel (\textit{host}).

Prostředníkem mezi hostem a hostitelem je hypervizor, což je software nebo firmware, který dovoluje virtualizovat systémové prostředky \cite{hypervisor}.
Existují dva typy hypervizorů.
Hypervizor typu 1 běží přímo nad hardwarem zatímco hypervizor typu 2 běží nad hostitelským OS, který řeší služby potřebné k virtualizaci samotné, jako například přístup ke I/O zařízením, správa paměti a další.

Nevýhodou plné virtualizace je nemožnost měnit množství přidělených prostředků (velikost RAM, množství CPU) virtuálnímu stroji za jeho běhu.

Mezi hypervizory patří například VMware, KVM a z/VM.

\subsubsection{Softwarová virtualizace (emulace)}

Softwarová virtualizace spoléhá na úpravu strojových instrukcí za běhu hostovaného systému.
Jedná se například o překlad privilegovaných instrukcí na neprivilegované a úpravy přístupů do paměti.
Z těchto důvodů je tento typ plné virtualizace není příliš efektivní.

\subsubsection{Hardwarově asistovaná virtualizace}

Jedná se o virtualizaci, která je podpořena procesorem hostitelského systému.
Použití hardwarově asistované virtualizace přináší oproti softwarové virtualizaci značné snížení režie.
Na CPU architektuře x86 se jedná o technologie VT-x (Intel) a AMD-V (AMD).
Princip spočívá v rozšíření instrukční sady CPU o další instrukce, které procesoru dovolují přepínat mezi více kontexty.
Tento přístup dovoluje provozovat nemodifikované instance OS uvnitř virtuálního prostředí.

\subsection{Paravirtualizace}

Paravirtualizace je technika, při které je hostovaný systém upraven tak, aby místo namísto přímé komunikace s hardwarem používal API hostitele.
Výhodou tohoto přístupu je snížená režie oproti plné softwarové virtualizaci.
V současné době se paravirtualizace používá v kombinaci s hardwarově asistovanou virtualizací, například přímé síťové komunikaci mezi hostem a hostitelem \cite{virtio}.


\subsection{Virtualizace na úrovni operačního systému}

Virtualizace na úrovni operačního systému je virtualizační technika, která dovoluje koexistenci několika oddělených uživatelských prostorů, které fungují nad stejným prostorem jádra.
Izolovaný uživatelský prostor se často označuje jako tzv. \uv{kontejner}.

Systémová paměť operačních systémů je dělena na dvě části: prostor jádra a uživatelský prostor \cite{memory_spaces}.
Prostor jádra je využíván pro běh jádra operačního systému a jeho služeb.
V~uživatelském prostoru běží procesy spuštěné uživatelem, bez přístupu do prostoru jádra.
Uživatelské procesy komunikují s~prostorem jádra pouze skrze systémová volání jádra.

Výhodou virtualizace na úrovni OS je menší režie oproti plné virtualizaci, protože už není nutné spouštět celou novou instanci operačního systému.
Oproti plné virtualizaci je naopak nutné zajistit, aby se jednotlivé uživatelské režimy nemohly neoprávněně ovlivňovat.
Toto je netriviální úloha a proto je vhodné zavést některá bezpečností opatření.
Jedním z~opatření je vytvoření kontejneru pod neprivilegovaným uživatelem (ne-root, ne-administrátor), takže případný \uv{únik} z~kontejneru nezpůsobí škodu.

Jelikož kontejner sdílí stejný prostor jádra se svým hostitelem, tak není možné, aby byl kontejner založen na jiném OS než hostitel.
Například není možné provozovat Linuxové kontejnery na systému s OS Windows.
Mezi technologie pro tvorbu kontejnerů patří například jmenné prostory a kontrolní skupiny (Linux), jails (BSD).

\subsubsection{Nástroje operačního systému Linux}

Linux jádro samo o sobě nenabízí přímé kontejnerové řešení, ale pouze sadu nástrojů, pomocí kterých je kontejner vytvořen.
Jedná se o jmenné prostory (\textit{namespaces}), kontrolní skupiny (\textit{cgroups}) a tzv \uv{capabilities}.
Nad těmito nástroji je postaveno například kontejnerové řešení LXC nebo Docker.

Velmi zjednodušeně se jedná o založení nového kořene adresářové struktury (Linux příkaz \verb|chroot|), následným spuštěním kontejneru pracující nad touto adresářovou strukturou a kontrolou přístupu kontejneru ke zdrojům hostitelského systému.
Vzory této adresářové struktury se nazývají obrazy (\textit{images}).

\subsubsection{Jmenné prostory (namespaces)}

Jmenné prostory \cite{namespaces} jsou nástroj Linuxového jádra sloužící k~vytvoření izolovaných skupin procesů.
Každý proces má svůj jmenný prostor, který mu vymezuje viditelnost pouze na vlastní prostředky a prostředky podřízených jmenných prostorů.
Jmenné prostory tvoří hierarchickou stromovou strukturu.
Při startu systému existuje nejméně jeden jmenný prostor.
Existuje několik druhů jmenných prostorů.

\paragraph{PID}

Proces vidí procesy pouze ze svého a podřízených PID jmenných prostorů.

\paragraph{NET}

Síťové rozhraní náleží právě do jednoho jmenného prostoru a není sdíleno s~podřízenými jmennými prostory.
Každé síťové rozhraní má své IP adresy, routovací tabulky, firewall a další síťové prostředky. 

\paragraph{MNT}

Kontrolujeme jaká připojená zařízení (\textit{mount point}) budou viditelné.
Vytvořením nového MNT jmenného prostoru zdědíme všechna připojená zařízení rodiče.
Změny v~aktuálním jmenném prostory neovlivňují jeho rodiče.

\paragraph{IPC}

Komunikace mezi procesy.
Linuxové jádro ve výchozím stavu přiděluje sdílenou paměť na základě uživatelského ID, což by vedlo ke kolizím mezi jmennými prostory.

\paragraph{UTS}

Umožňuje systému vystupovat pod více hostitelskými jmény (\textit{hostnames)}.

\paragraph{user}

Umožňuje libovolně mapovat uživatelská ID z~kontejneru na ID uživatelů z~hostitele.
Uživatel v~kontejneru se pak může tvářit jako root i když vně kontejneru se jedná pouze o~obyčejného uživatele (bez root práv).

% https://www.youtube.com/watch?v=sK5i-N34im8

O~vytvoření jmenného prostoru se starají systémová volání \verb|fork|, \verb|unshare| a \verb|clone|.

\subsubsection{Kontrolní skupiny (cgroups)}

Kontrolní skupiny jsou nástroj Linuxového jádra umožňující přidělovat skupinám procesů limit na systémové prostředky a prioritizovat jednotlivé skupiny při přístupu k~nim \cite{cgroups}.
Mezi typy prostředků patří například:

\begin{itemize}
	\item CPU čas,
	\item operační paměť,
	\item síťový provoz,
	\item přístup na disk.
\end{itemize} % konkretni nazev, brat z man stranky linuxu

\subsubsection{Capabilities}

Historicky UNIXové systémy dělily procesy na privilegované a neprivilegované.
Privilegované procesy pak nepodléhaly klasické kontrole oprávnění.
Nicméně pokud proces potřebuje pouze některá oprávnění, tak není vhodné z bezpečnostního hlediska z něj udělat proces privilegovaný, ale pouze mu přidělit oprávnění, která vyžaduje.
Toto dělení oprávnění je zavedeno do systému pomocí tzv. \uv{capabilities} \cite{capabilities}.

Například příkaz \verb|ping|, využívající protokol ICMP, potřebuje přímý přístup k socketu a tím pádem by privilegovaný.
S využitím \textit{capabilities} je možné tomuto procesu přidělit pouze oprávnění \verb|CAP_NET_RAW|.

\subsubsection{Copy on write (CoW)}

Copy on write je technika optimalizace správy dat, při které se při kopírování dat namísto vytvoření jejich kopie pouze označí jako sdílená \cite{cow}.
Fyzickou kopii dat je třeba vytvořit až v~okamžiku jejich modifikace.
Za cenu vyšší režie můžeme takto ušetřit mnoho paměti.

V kontextu kontejnerů lze CoW využít k jejich velmi rychlému vytváření (vytvoření kopie obrazu) nebo k vytváření tzv. \uv{snapshotů}.

\subsection{BSD jails}

BSD jails je kontejnerová technologie nacházející se v operačních systémech typu BSD \cite{jails}.
Na rozdíl od Linuxových kontejnerů, kde je konečný kontejner poskládán z jednotlivých nástrojů, nám BSD nabízí přímo sadu systémových volání (\verb|jail_*|).
Kontejner nám zajišťuje izolaci:

\begin{itemize}
	\item Proces vidí procesy ze svého kontejneru.
	\item Modifikace síťového nastavení z kontejneru není povolena (pevná IP adresa, routovací tabulky).
	\item Kontejner nemá přímý přístup k síťovému socketu (ICMP protokol nebude fungovat).
	\item Kontejner je omezen pouze na adresářovou strukturu, ve které byl založen.
\end{itemize}

Kontejneru lze také omezovat systémové prostředky (RAM, disk, síť, \ldots).
