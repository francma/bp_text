\section{GitLab CI}

GitLab CI je systém kontinuální integrace, který je součástí správce Git repozitářů a plánovače projektů GitLab.
Jeho použití je tedy omezena pouze pro uživatele tohoto systému.
Projekt je dostupný pod open-source licencí (komunitní edice) a komerční \cite{gitlab_ce}. 

\subsection{Průběh integrace}

% shadow tasky, promenne env a podminene spusteni

Konfigurace integrace je řešena pomocí souboru \verb|.gitlab-ci.yml|, který je umístěn přímo v~kořenovém adresáři Git repozitáře projektu.

Proces integrace (\textit{pipeline}) je rozdělen na fáze (\textit{stages}), úkoly (\textit{jobs}) a příkazy.
Fáze se vykonávají sekvenčně za sebou v~zadaném pořadí.
Pokud jedna z~fází selže, tak celý proces integrace končí chybou.

\begin{listing}[ht]
\caption{Definice fázi v .gitlab-ci.yml}
\begin{minted}[frame=single,linenos]{yaml}
stages:
  - test
  - build
  - deploy
\end{minted}
\end{listing}

Jednotlivé fáze se skládají z~několika úkolů, které jsou v rámci fáze spouštěny nezávisle na sobě a tím pádem je lze paralelizovat.
Selhání jednoho z~úkolů vede k~neúspěchu celé fáze.

\begin{listing}[ht]
\caption{Definice úkolu v .gitlab-ci.yml}
\begin{minted}[frame=single,linenos]{yaml}
job1:
  stage: test
  script:
    - cmd 1
    - cmd 2
\end{minted}
\end{listing}

Každý úkol je složen z~několika příkazů.
Pokud jeden z~příkazů skončí chybovým návratovým kódem, pak je ukončeno provádění úkolu.

Mezi speciální úkoly patří \verb|before_script| a \verb|after_script|.
Tyto úkoly jsou provedeny vždy před nebo po každém uživatelsky definovaném úkolu.

\begin{listing}[ht]
\begin{minted}[frame=single,linenos]{yaml}
before_script:
  - cmd 1
  - cmd 2

after_script:
  - cmd 1
  - cmd 2
\end{minted}
\caption{Definice before\_script a after\_script .gitlab-ci.yml}
\end{listing}

Jednotlivé úkolu lze podmíněně spouštět pomocí notace \verb|when|.

\begin{itemize}
  \item \verb|on_success| -- vykonej pouze pokud předchozí fáze skončila úspěchem (výchozí) 
  \item \verb|on_failure| -- vykonej pouze pokud předchozí fáze skončila neúspěchem
  \item \verb|always| -- vykonej vždy
\end{itemize}

\subsection{Architektura}

Systém GitLabu je rozdělen do několika podsystémů, z nichž GitLab CI Runner je zodpovědný za vykonání úkolů integrace.
Integrační proces je pak korigován GitLab systémem.
\cite{gitlab_architecture}

\subsubsection{GitLab CI runners}

CI systém spolupracuje s~několika běhovými servery (\textit{Runners}), které se starají o~samotné vykonání integrace.
Běhový server je schopen provozovat paralelně několik běhových prostředí různého typu.
Volba běhového serveru je definována na úrovni konfigurace projektu a je omezena globálním nastavením GitLab CI.

% https://docs.gitlab.com/ee/ci/runners/README.html
% http://docs.gitlab.com/ce/api/ci/builds.html

\paragraph{Shell běhové prostředí}

Příkazy provádí na stejném systému pod stejným uživatelem jako samotný běhový server.
Z toho plyne mnoho bezpečnostních rizik, a proto se použití tohoto běhového prostředí nedoporučuje v produkčním nasazení.

\paragraph{Docker běhové prostředí}

Příkazy se provádí uvnitř Docker kontejneru.
Volba Docker obrazu je definována klíčovým slovem \verb|image|.
Integraci lze obohatit o~tzv. služby, kterými můžeme každému úkolu dát k~dispozici přístup k~další Docker instanci, která bude dostupná pod zadanou hostname.
Služby lze opět konfigurovat jak na úrovni integrace, tak i na úrovni úkolu.

\begin{listing}[ht]
\begin{minted}[frame=single,linenos]{yaml}
image: ruby:2.2

services:
  - postgres:9.3

test:
  script:
  - bundle exec rake spec
\end{minted}
\caption{Definice služeb v .gitlab-ci.yml}
\end{listing}

\paragraph{VirtualBox a Parallels}

Běhová prostředí stavějící na plné virtualizaci.
Virtuální stroj musí podporovat Bash kompatibilní shell a být dostupný přes SSH spojení.
Virtuální stroje jsou klonovány pro udržení čistého prostředíaaa a snapshotovány pro urychlení integrací.

% https://docs.gitlab.com/runner/executors/README.html

\subsubsection{Cache}

% artefakty vs cache
% cache by se nemela pouzivat k predani

Pro urychlení integrace se používá cache.
Například je takto možné sdílet závislosti projektu mezi různými integracemi.
Ve výchozím stavu je cache sdílena mezi \textit{branch} a \textit{job}.
Toto chování lze upravit nastavením klíče, který bude identifikovat danou cache.
Při tvorbě klíče lze využít předdefinované CI proměnné. 
Soubory a složky určené k~cachování uvedeme jako výčet v~konfiguraci projektu.
Uvedené cesty jsou uvedeny relativně ke kořeni projektu, cachování souborů mimo projekt není touto cestou možné. % prapodivna veta 
Cache lze také omezit pouze na soubory a složky, které nejsou verzovány Gitem.

\begin{listing}[ht]
\begin{minted}[frame=single,linenos]{yaml}
cache:
  key: "$CI_BUILD_NAME/$CI_BUILD_REF_NAME"
  untracked: true
  paths:
    - binaries/
    - .config
\end{minted}
\caption{Definice cache v .gitlab-ci.yml}
\end{listing}

% S3 api zabudovane

Cache je ve výchozím nastavení ukládána přímo na běhovém serveru.
Pokud fáze běží na různých běhových serverech, tak se cache nesdílí.
Alternativně můžeme použít sdílené cachování na S3 kompatibilních serverech.
S3 je REST API, které bylo původně vytvořeno jako část Amazon services.
Toto API implementuje samotný Amazon, tak i nějaké open-source projekty, které lze provozovat na vlastní infrastruktuře.

% https://gitlab.com/gitlab-org/gitlab-ci-multi-runner/merge_requests/88
% https://docs.aws.amazon.com/AmazonS3/latest/API/Welcome.html
% https://github.com/minio/minio/

\subsection{Uživatelské rozhraní}

% https://about.gitlab.com/applications
GitLab umožňuje přístup k informacím systému pomocí řady aplikací, které využívají GitLab HTTP API.

\subsubsection{API}

GitLab dovoluje přístup ke zdrojům systému včetně zdrojů spjatých s integrací pomocí HTTP API.
K autentizaci uživatele se používá tokenů, které jsou součástí HTTP hlaviček nebo HTTP query parametrů.
K reprezentaci dat API používá formátu JSON (viz. ukázka \ref{code:gitlab-api-json-response}).

\begin{table}[htbp]
\centering
\fontsize{9.5}{11.5}\selectfont
\caption{GitLab HTTP API}
\label{table:gitlab-api-spec}
\begin{tabular}{|l|l|l|}
\hline
HTTP metoda + akce                                          & Popis \\ \hline
\verb|GET  /projects/{id}/jobs|                             & Vrátí seznam úkolů projektu      \\ \hline
\verb|GET  /projects/{id}/pipelines/{pipeline_id}/jobs|     & Vrátí seznam úkolů integrace      \\ \hline
\verb|POST /projects/{id}/jobs/{job_id}/cancel|             & Zruší probíhající úkol      \\ \hline
\verb|POST /projects/{id}/jobs/{job_id}/retry|              & Restartuje úkol      \\ \hline
\verb|POST /projects/{id}/jobs/{job_id}/erase|              & Smaže úkol      \\ \hline
\verb|POST /projects/{id}/pipeline|                         & Vytvoří novou integraci      \\ \hline
\verb|GET  /projects/{id}/pipelines/{pipeline_id}|          & Vrátí integraci      \\ \hline
\verb|POST /projects/{id}/pipelines/{pipeline_id}/cancel|   & Zruší probíhající integraci      \\ \hline
\verb|POST /projects/{id}/pipelines/{pipeline_id}/retry|    & Restartuje probíhající integraci      \\ \hline
\end{tabular}
\end{table}

Seznam API zdrojů v tabulce \ref{table:gitlab-api-spec} je pouze výňatek, úplný seznam zdrojů je k dostání v oficiální dokumentaci \cite{gitlab_api}. 
Fáze integrace nemá přidělený svůj vlastní API koncový bod.

\begin{listing}[ht]
\caption{\label{code:gitlab-api-json-response}Odpověď GitLab API (detail integrace)}
\begin{minted}[frame=single,linenos]{json}
{
  "id": 46,
  "status": "success",
  "ref": "master",
  "sha": "a91957a858320c0e17f3a0eca7cfacbff50ea29a",
  "before_sha": "a91957a858320c0e17f3a0eca7cfacbff50ea29a",
  "tag": false,
  "created_at": "2016-08-11T11:28:34.085Z",
  "updated_at": "2016-08-11T11:32:35.169Z",
  "finished_at": "2016-08-11T11:32:35.145Z",
  "coverage": "30.0"
}
\end{minted}
\end{listing}

\subsubsection{CLI rozhraní}

GitLab neposkytuje přímý terminálový přístup ke svým zdrojům.
Místo toho lze využít neoficiálních aplikací, které vytváří cli \uv{obálku} nad již zmíněným HTTP API.
Autentizace je řešena pomocí GitLab tokenu, který se nastaví jako \verb|GITLAB_API_PRIVATE_TOKEN| proměnná prostředí.

% https://github.com/NARKOZ/gitlab

\begin{listing}[ht]
\begin{minted}[frame=single,linenos]{shell}
# Vrať seznam úkolů projekty s ID = 1
gitlab jobs 1
# Vrať integraci z projektu 1 s ID = 2
pipeline 1 2
\end{minted}
\label{code:gitlab-api}
\caption{Odpověď GitLab API (detail integrace)}
\end{listing}

\subsubsection{Webové uživatelské rozhraní}

Součástí GitLabu je i webové uživatelské rozhraní.
Najdeme zde informace jak o Git repozitářích, tak i integracích.
Součástí webového rozhraní je sledování stavu integrací v reálném čase.

\imagefigurelarge{gitlab-webui.png}{Ukázka GitLab web UI}

\subsection{Možnosti instalace}

GitLab nabízí několik instalačních metod.
První z nich jsou klasické Linuxové distribuční balíčky (apt, yum, \ldots).
Další možností je využití připravených obrazů pro kontejnerové technologie (Docker, \ldots).
GitLab nám také dává možnost provozovat systém na jejich vlastní infrastruktuře.

% https://about.gitlab.com/installation/
