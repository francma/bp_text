\section{Verzovací nástroje}

Verzovací nástroj (VCS) zaznamenává historii změn souborů v repozitáři tak, aby bylo možné se kdykoliv vrátit ke specifické verzi souboru \cite{pro_git}.
Například i tato bakalářská práce je psána s podporou verzovacího nástroje.

\subsection{Spolupráce VCS Git s CI}

Příklad spolupráce mezi VCS a CI ukážeme na verzovacím systému Git.
Git používá distribuovaný model, což znamená, že repozitář je nejen uložen na hlavním serveru (\textit{origin}), ale také na klientských stanicích.

Git pracuje s větvemi (\textit{branches}), které se skládají ze snímků (\textit{commits})
Větve jde slučovat mezi sebou do sebe.
Větev s názvem \textit{master} je označena jako hlavní a její poslední verze často reprezentuje produkční verzi projektu.

\begin{listing}[ht]
\caption{\label{vcs-and-ci}Git v kombinaci s CI}
\begin{minted}[frame=single,linenos]{shell}
# Stáhneme master větev hlavního server 
git clone https://github.com/francma/piper-ci-core .
# Vytvoříme novou větev, vycházející z master větve
git checkout -b shell
# Provedeme změny (...)
# Vytvoříme nový snapshot a nahrajeme změny na hlavní server
git commit -am 'Implementace shellu' && git push origin shell
# Po úspěšném dokončení testů nahrajeme naše změny do hlavní větve
git checkout master && git merge shell && git push origin master
\end{minted}
\end{listing}

Pro pohodlnější správu Git repozitářů můžeme využít webové uživatelské rozhraní, například GitHub.
Mimo grafické reprezentace snímků (\textit{commits}) GitHub také podporuje tzv. \uv{pull request}. 
Pull request nahrazuje poslední krok (\verb|git push|) z ukázky kódu \ref{vcs-and-ci} a přináší přívětivou uživatelskou obálku.

\imagefigurefull{github-mr.png}{Ukázka GitHub pull requestu}{0.8}
