\begin{introduction}

Programování a procesy s~ním spojené prochází neustálým vývojem, který tuto disciplínu stále mění. 
Přidáváním vyšších vrstev abstrakce se vzdalujeme od stroje samotného a přecházíme k~simulaci fungování reálného světa, čímž se stává vývoj softwaru neustále rychlejším a dynamičtějším. 

Jsou to také nově vznikající vývojové techniky, které urychlují vývoj softwaru.
Jednou z~nich je i technika kontinuální integrace (zkráceně CI).
Tato technika spočívá v~časté integraci kódu mezi vývojáři ve společném repozitáři a následným testováním. 
Mezi testy například patří:

\begin{enumerate}
	\item Integrační a jednotkové testy kontrolující funkčnost programu
	\item Testy kontrolující jednotný styl kódu v~projektu
	\item Syntaktické testy
\end{enumerate}

Kontinuální integrace je úzce spjatá s~verzovacím systémem, který spravuje repozitář projektu.
Samotný systém kontinuální intgrace, lze definovat jako množinu softwaru realizující následující úkony:

\begin{enumerate}
	\item Naslouchat a reagovat na změny v~repozitáři projektu
	\item Připravit běhové prostředí vhodné pro testovací proces
	\item Stáhnout zdrojové soubory projektu z~repozitáře
	\item Sestavit stážený projekt včetně jeho závislostí
	\item Spustit testovaní dle zadání projektu
	\item Informovat vývojáře a další systémy o~výsledku testování
\end{enumerate}

\end{introduction}
