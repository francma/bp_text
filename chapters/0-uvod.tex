\begin{introduction}

Programování a procesy s~ním spojené prochází neustálým vývojem, který tuto disciplínu stále mění. 
Přidáváním vyšších vrstev abstrakce se vzdalujeme od stroje samotného a přecházíme k~simulaci fungování reálného světa, čímž se stává vývoj softwaru neustále rychlejším a dynamičtějším. 

Jsou to také nově vznikající vývojové techniky, které urychlují vývoj softwaru.
Jednou z~nich je i technika kontinuální integrace (zkráceně CI).
Tato technika spočívá v~časté integraci kódu mezi vývojáři ve společném repozitáři.
Během integrace kódu se mohou provádět i různé testy, například:

\begin{enumerate}
	\item Integrační a jednotkové testy kontrolující funkčnost programu.
	\item Testy kontrolující jednotný styl kódu v~projektu.
	\item Syntaktické testy.
\end{enumerate}

Kontinuální integrace je úzce spjatá s~verzovacím systémem, který spravuje repozitář projektu.
Samotný systém kontinuální integrace, lze definovat jako množinu softwaru realizující následující úkony:

\begin{enumerate}
	\item Naslouchat a reagovat na změny v~repozitáři projektu.
	\item Připravit běhové prostředí vhodné pro testovací proces.
	\item Stáhnout zdrojové soubory projektu z~repozitáře.
	\item Sestavit stažený projekt včetně jeho závislostí.
	\item Spustit testovaní dle zadání projektu.
	\item Informovat vývojáře a další systémy o~výsledku testování.
\end{enumerate}

\end{introduction}

Tento popis je velmi obecný a proto bude vhodné se podívat na nějaký konkrétní projekt používající kontinuální integraci.
Vezmeme například projekt Nette.
Nette je PHP framework pro stavbu webových aplikací.
Repozitáře projektu jsou verzované verzovacím nástrojem Git a další integrace je pod systémem kontinuální integrace Travis CI.

\begin{enumerate}
	\item Spusť integrační a jednotkové testy pro verze PHP 7.1 a 7.2 pro terminálové rozhraní (php-cgi) a webové (php).
	\item Zkontroluj syntakticky správný zápis programu.
	\item Předej informace o testovaní službě třetí strany.
\end{enumerate}

Z tohoto nám vznikne 6 úkolů:

\begin{itemize}
	\item Kontrola syntaxe.
	\item Testy na php 7.1.
	\item Testy na php 7.2.
	\item Testy na php-cgi 7.1.
	\item Testy na php-cgi 7.2.
	\item Předání informací o testování službě třetí strany.
\end{itemize}

Jelikož nemá smysl pouštět jakékoliv testování, pokud program není zapsán syntakticky správně, tak jako první provedeme kontrolu syntaxe.
Po zkontrolovaní syntaktické správnosti spustíme integrační a jednotkové testy.
Protože testy na sobě nijak nezávisí, tak je možné je pustit paralelně, abychom zkrátili dobu testování.
Po dokončení testování předáme výsledky službě třetí strany.

Skupiny testů, které jsou na sobě nezávislé a je možné je paralelizovat nazýváme fází.
V tomto konkrétním případě nám vzniknou 3 fáze:

\begin{itemize}
	\item Kontrola syntaxe.
	\item Testy.
	\item Předání informací o testování službě třetí strany.
\end{itemize}

Od systému kontinuální integrace tedy očekáváme určitou možnost integrační proces řídit.
Většinou se jedná o rozdělení na fáze a úkoly, ale také o další možnosti, které budou probrány u konkrétních systémů CI.

Uživatele CI systému také zajímá výstup z jeho zadaných příkazů.
Velmi vhodné je to například při hledání chyby, kdy by pouhá informace o neúspěchu integrace byla nedostačující.
Mnoho CI systémů implementuje tzv. \uv{streaming logy}, což znamená, že výstup z příkazů je dostupný v \uv{reálném čase} místo nutnosti čekat na doběhnutí celé integrace.

Cílem této práce je navrhnout minimalistický systém kontinuální integrace, který bude určen pro menší komunitní projekty.
Hlavní vlastností výsledného projektu bude jeho modulárnost.

V této práci se nejprve budeme zabývat technologií vytváření izolovaných běhových prostředí, jelikož jsou základním stavebním kamenem systému kontinuální integrace.
V další části se podíváme na existující systémy kontinuální integrace a následně navrhneme systém vlastní.
Dále bude následovat samotná implementace a testování vytvořeného systému.

