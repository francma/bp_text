\chapter{Implementace}

Piper~CI je rozdělen do několika částí, subsystému.
Tím pádem je možné libovolnou jeho část nahradit jinou, nebo systém takto rozšířit o další funkcionalitu.
Jednotlivé části systému spolu budou komunikovat skrze definované API.
Jádro musí také implementovat uživatelské účtu, jejich autentizaci a autorizaci při přístupu ke svým zdrojům.

Jako hlavní programovací jazyk jsem zvolil Python ve verzi 3.6, protože s ním mám už předchozí zkušenosti.
Také je to jediný programovací jazyk ze zadaných (Rust, Lua, Ruby, Python), který se vyučuje v rámci bakalářského studia na naší fakultě.
Rozhodl jsem se pro jeho poslední verzi z důvodu podpory typových anotací, které zlepšují čitelnost kódu a při vhodném pojmenování tříd dokáží částečně nahradit dokumentační komentáře.

\begin{listing}[ht]
\begin{minted}[frame=single,linenos]{python}
def foo(bar: Bar, count: int) -> List[str]:
    items: List[str] = ['hello:' + i for i in bar.items()]
    return items
\end{minted}
\caption{Ukázka typových anotací v Pythonu 3.6}
\end{listing}

\section{Jádro (Piper CI core)}

Středobod celeho systému.
Stará se o persistenci dat, která poskytuje v rámci HTTP REST API.
REST API jsem zvolil z důvodu jeho jednoduchosti na implementaci ze strany klientů i serveru.
Jeho dalším úkolem je komunikace s běhovými klienty, kterým přiděluje práci a ukládá výsledky provedených integrací.

K persistenci dat jsem použil knihovnu peewee, která mi poskytuje abstrakci nad různými databázovými stroji.
Abstrakcí jsem ztratil možnost optimalizace pro konkrétní databázový stroj, ale naopak získal možnost provozu systému nad různými databázovými stroji včetně SQLite.
Jádro systému nepotřebuje kromě jednoduché filtrace dat dělat náročnější dotazy do databáze, tak případný ztracený výkon nebude ani znatelný.

Volba webové knihovny padla na webový framework Flask.
Flask je dostatečně lehký, aby splńoval požadavek na minimalismus systému.
Flask v sobě nese i webový server, který je ale vhodný jen pro testovací účely.
Produkční provoz se doporučuje v kombinaci s wsgi implementujícím serverem a volitelně i reverzní proxy (například nginx).

Jádro s běhovými klienty komunikuje pomocí REST API.
Jádro nebere zodpovědnost za přidělování jednotlivých úkolů běhovým klientům, ale pouze naslouchá a čeká na klienty až se dotáží jádra.
Tímto je zaručeno, že běhový klient si vezme jen tolik úkolů, kolik je schopen zvládnout.
Úkoly jsou uložené v FIFO frontě.

Součástí jádra je i terminálové SSH rozhraní, ke kterému se uživatel autentizuje pomocí RSA klíčů.
REST rozhrání používá k autentizaci tokenů, které jsou součástí HTTP hlaviček dotazů na server.

\section{Běhový klient (Piper CI LXD runner)}

Běhový klient je zodpovědný za konečný průběh integrací.
Kontrétně se jedná o vykonání zadaných příkazů a předání výstupu těchto příkazů jádru systému, které je dalé zpracuje.
Běhové prostředí je implemtováno technologií LXD.
Ke komunikaci mezi Pythonem a LXD jsem použil oficiální pylxd knihovnu.

Běhový klient převezmu úkol jako sadu příkazů, které sloučí do sebe a tím vytvoří spustitelný skript pro běhové prostředí.
Výstup z tohotu skriptu je přenášen jádru systému pomocí periodických dotazů na REST api jádra.
Výstup tedy není \uv{streamovaný} v realném čase, ale nabírá zpoždění v řádu jednotek sekund.
Toto ale není problém, protože dokončení úkolu trvá průměrně několik minut a na výstup z něho není třeba žádné okamžité reakce.
Aby bylo možné ve výstupu identifikovat jednolivé příkazy, tak je skript opatřen speciálními značkami, které udávají začátek a konec příkazu s dalšími informacemi jako například návratový kód a aktuální čas.
Tyto značky jsou následně rozpoznáni jádrem systému, které je dále zpracuje.

Běhový klient je schopen v rámci jedné své instance spouštět více úkolů paralelně.
Paralelizaci jsem implementoval pomocí knihovny multiprocessing.

\section{Webové rozhraní (Piper CI web)}

Webové rozhraní využívající REST API jádra pro zobrazení informací o stavů integrací.
K autorizaci se používá již zníněný uživatelský token.
Streaming logů je implementován pomocí periodických dotazů (short polling) z důvodů zmíněných již dříve.
Před výpisem logů je potřeba nahradit ANSI sekvence ve výstupu příkazů HTML značkami.
To se děje na straně prohlížeče pomocí JavaScriptu.

