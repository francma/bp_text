\chapter{Implementace}

Za hlavní programovací jazyk byl zvolen Python ve verzi 3.6, protože s ním má autor této práce největší zkušenosti.
Také je to jediný programovací jazyk ze zadaných (Rust, Lua, Ruby, Python), který se vyučuje v rámci bakalářského studia na fakultě informačních technologií ČVUT.

Rozhodnutí padlo na poslední stabilní verzi Pythonu (3.6) bez podpory předchozích verzí hlavně z důvodu podpory typových anotací, které zlepšují čitelnost kódu a při vhodném pojmenování tříd a proměnných dokáží do jisté míry nahradit dokumentační komentáře.

\begin{listing}[ht]
\begin{minted}[frame=single,linenos]{python}
def foo(bar: Bar, count: int) -> List[str]:
    items: List[str] = ['hello:' + i for i in bar.items()]
    return items
\end{minted}
\caption{Ukázka typových anotací v Pythonu 3.6}
\end{listing}

\section{Jádro (Piper CI core)}

Volba webové knihovny padla na webový framework Flask.
Flask je dostatečně lehký, aby splńoval požadavek na minimalismus systému.
Flask v sobě nese i webový server, který je ale vhodný jen pro testovací účely, produkční provoz se doporučuje v kombinaci s wsgi implementujícím serverem a volitelně i reverzní proxy (například nginx).
Validace vstupních dat je prováděna proti OpenAPI schématu.

K persistenci dat byla využita knihovna peewee, která poskytuje abstrakci nad různými databázovými stroji.
Abstrakcí byla ztracena možnost optimalizace pro konkrétní databázový stroj, ale naopak získána možnost provozu systému nad různými databázovými stroji včetně SQLite.
Jádro systému nepotřebuje kromě jednoduché filtrace dat dělat náročnější dotazy do databáze, tak případný ztracený výkon nebude tak znatelný.

Logy úkolů jsou ukládány v textovém soubory, který je vázán s konkrétním úkolem identifikátorem v jeho názvu.

Jádro s běhovými klienty komunikuje pomocí REST API.
Jádro nebere zodpovědnost za přidělování jednotlivých úkolů běhovým klientům, ale pouze naslouchá a čeká na klienty až se dotáží jádra.
Tímto je zaručeno, že běhový klient si vezme jen tolik úkolů, kolik je schopen zvládnout.
Úkoly jsou uložené v FIFO frontě, kterou implemetuje Redis server.

REST rozhrání používá k autentizaci tokenů, které jsou součástí HTTP hlaviček dotazů na server.
Součástí jádra je i terminálové SSH rozhraní, ke kterému se uživatel autentizuje pomocí RSA klíčů.

\begin{listing}[ht]
\begin{minted}[frame=single,linenos]{text}
localhost$ ssh piper@piper.martinfranc.eu
Enter passphrase for key '/home/anon/.ssh/id_rsa': 
[email@martinfranc.eu]$ help

Documented commands (type help <topic>):
========================================
build  exit  help  identity  job  project  runner  stage  user

\end{minted}
\caption{Ukázka terminálového rozhraní}
\end{listing}

Terminálové rozhraní je implementováno pomocí knihovny \verb|cmd|, která dovoluje i psaní interaktivní dokumentace parsované z komentářů funkcí.

\section{Běhový klient (Piper CI LXD runner)}

Běhové prostředí je implemtováno technologií LXD.
Ke komunikaci mezi Pythonem a LXD je použita oficiální pylxd knihovna.

Běhový klient převezmu úkol jako sadu příkazů, které sloučí do sebe a tím vytvoří spustitelný skript pro běhové prostředí.
Výstup z tohotu skriptu je přenášen jádru systému pomocí periodických dotazů na REST api jádra.
Výstup tedy není \uv{streamovaný} v realném čase, ale nabírá zpoždění v řádu jednotek sekund.
Toto ale není problém, protože dokončení úkolu trvá průměrně několik minut a na výstup z něho není třeba žádné okamžité reakce.
Aby bylo možné ve výstupu identifikovat jednolivé příkazy, tak je skript opatřen speciálními značkami, které udávají začátek a konec příkazu s dalšími informacemi jako například návratový kód a aktuální čas.
Tyto značky jsou následně rozpoznáni jádrem systému, které je dále zpracuje.

\begin{listing}[ht]
\begin{minted}[frame=single,linenos]{text}
::piper:command:0:start:1514982032::
fetch http://alpine.org/alpine/v3.5/main/x86_64/APKINDEX.tar.gz
fetch http://alpine.org/alpine/v3.5/community/x86_64/APKINDEX.tar.gz
v3.5.2-237-g9e642dbbba [http://alpine.org/alpine/v3.5/main]
v3.5.2-213-g0860a961bc [http://alpine.org/alpine/v3.5/community]
OK: 7963 distinct packages available
::piper:command:0:end:1514982032:0::
::piper:command:1:start:1514982032::
(1/10) Installing libbz2 (1.0.6-r5)
(2/10) Installing expat (2.2.0-r1)
(3/10) Installing libffi (3.2.1-r2)
(4/10) Installing gdbm (1.12-r0)
(5/10) Installing ncurses-terminfo-base (6.0_p20170701-r0)
(6/10) Installing ncurses-terminfo (6.0_p20170701-r0)
(7/10) Installing ncurses-libs (6.0_p20170701-r0)
(8/10) Installing readline (6.3.008-r4)
(9/10) Installing sqlite-libs (3.15.2-r1)
(10/10) Installing python2 (2.7.13-r0)
Executing busybox-1.25.1-r1.trigger
OK: 53 MiB in 26 packages
::piper:command:1:end:1514982033:0::
::piper:command:2:start:1514982033::
Hello!
::piper:command:2:end:1514982033:0::
\end{minted}
\caption{Ukázka výstupu logu}
\end{listing}

Běhový klient je schopen v rámci jedné své instance spouštět více úkolů paralelně.
K paralelizaci je použita knihovna multiprocessing.

\section{Webové rozhraní (Piper CI web)}

Webové rozhraní využívající REST API jádra pro zobrazení informací o stavů integrací.
K autorizaci se používá již zníněný uživatelský token.
Streaming logů je implementován pomocí periodických dotazů (short polling).
Před výpisem logů je potřeba nahradit ANSI sekvence ve výstupu příkazů HTML značkami.
To se děje na straně prohlížeče pomocí JavaScriptu.

\imagefigurelarge{piper-web.png}{Výsledné webové rozhraní}

