\chapter{Implementace}

\section{Architektura}

Piper~CI bude rozdělen do několika částí, subsystému.
Tím pádem bude možné libovolnou jeho část nahradit jinou, nebo systém takto rozšířit o další funkcionalitu.
Jednotlivé části systému spolu budou komunikovat skrze definované API.
Jádro musí také implementovat uživatelské účtu, jejich autentizaci a autorizaci při přístupu ke svým zdrojům.

\subsection{Jádro}

Středobod celeho systému.
Stará se o persistenci dat, která poskytuje v rámci svého API.
Jeho dalším úkolem je komunikace s běhovými klienty, kterým přiděluje práci a ukládá výsledky provedených integrací.

\subsection{Běhový klient}

Běhový klient je zodpovědný za konečný průběh integrací.
Kontrétně se jedná o vykonání zadaných příkazů a předání výstupu těchto příkazů jadru systému, které je dalé zpracuje.
Běhový klient by měl využít vlastností běhových prostředí.
Je na místě, aby byl běhový klient schopen spouštět více souběžných integrací.

\subsection{Webové rozhraní}

Webové rozhraní využívající API jádra pro zobrazení informací o stavů integrací.

\subsection{Terminálové rozhraní}

Podobně jako webové rozhraní jen pomocí terminálu s využitím technologie SSH.

\imagefigurelarge{architektura_piper.pdf}{Návrh architektury}