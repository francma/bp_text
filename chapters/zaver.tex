\chapter{Závěr}

Výsledkem této práce je minimalistický CI systém vhodný pro menší komunitní projekty.

Práce po úvodu do problematiky kontinuální integrace vysvětluje technologie pro vytváření běhových prostředí a verzovacích nástrojů, jelikož jsou to základní stavební kameny kontinuální integrace.
Dále následuje analýza existujících řešení.
Konkrétně Gitlab CI, Travis CI a Buildbot.

Po analýze následuje vlastní návrh a následná implementace.

Minimalismu je docíleno pečlivým výběrem závislostí třetích stran.
Konkrétně je pro provoz systému nutný pouze Python verze 3.6 a vyšší, Redis server a technologie pro vytvoření běhového prostředí.

Modularitu a rozšiřitelnost systému je zaručeno rozdělením systému do dílčích systému s rozdělenou zodpovědností a přístupy ke zdrojům pomocí REST API.
Systém je tedy možné rozšířit o libovolnou podporu běhového prostředí nebo naimplementovat vlastní uživatelské rozhraní.
Součástí práce je implementace běhového prostředí využívající technologii LXD.
REST API dovoluje také snadnou integraci na další systémy.

Paralelizace je dosaženo rozdělením integrace do úkolů, které jsou přiřazovany do fází.
Úkoly v rámci fáze je pak možné provozovat paralelně.
Výsledné řešení je odlišně od zadaného tzv. \uv{build matrixu}, protože během rešeršní části se ukázal jako vhodnější přístup dělení na fáze a úkoly.

Součástí práce je i implementace webového rozhraní a terminálového rozhraní pomocí SSH.
Pomocí terminálového i webového rozhraní je možné sledovat stav úloh i včetně tzv. \uv{streaming logů}.

Součástí textu práce je také postup nasazení a uživatelská dokumentace.

V budoucnu bych rád tento projekt dále rozšiřoval.
Například bych rád projekt rozšířil o podporu dalších běhových prostředí a podíval se na možnosti instlace projektu pomocí Snap balíčků.



