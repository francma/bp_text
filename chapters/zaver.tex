\chapter{Závěr}

Cílem práce bylo navrhnout a implementovat minimalistický CI systém vhodný pro menší komunitní projekty.
Vytvořený systém splňuje požadavek na modulární podporu běhových prostředí rozdělením systému do více částí, které spolu mezi sebou komunikují pomocí HTTP API.
Systém aktuálně podporuje běhové prostředí založené na LXC/LXD kontejnerech.
Implementací HTTP API bylo docíleno také snadné rozšiřitelnosti a integrace na další systémy.

Sledování stavu integrací a správu projektů je možné přes terminálové SSH i webové rozhraní, včetně tzv. \uv{streaming logu}.
Integrace lze paralelizovat rozdělením na části (úkoly), které jsou v rámci své fáze spouštěny paralelně.

Vývoj výsledného systému i psaní tohoto textu bylo podpořeno systémem kontinuální integrace Travis CI.

V textu práce se autor věnuje kontinuální integraci a technik s ní spojených (testování softwaru, verzovací systémy, běhová prostředí) a analýze existujících řešení.

Součástí příloh této práce je také uživatelská dokumentace.

\section{Přínosy práce}

Výsledný systém oproti analyzovaným řešením přináší přihlašování uživatelů do systému pomocí asymetrických klíčů, které jsou z hlediska bezpečnosti lepší než klasické jméno a heslo.
Systém také implementuje zatím málo rozšířené LXD běhové prostředí.
Zdrojový kód práce je uvolněn pod open-source licencí a zveřejněn na GitHubu.

\section{Možnosti dalšího rozvoje}

Autor práce by v budoucnu rád rozšířil podporovaná běhová prostředí o další technologie (například VirtualBox) a podíval se na možnosti instalace systému pomocí balíčkovacích systémů (aports, \ldots).