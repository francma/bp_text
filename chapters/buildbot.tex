\section{Buildbot}

Buildbot je framework kontinuální integrace, který nabízí velké množství flexibility v konfiguraci integračního procesu \cite{buildbot}.
Oproti službám CI nám framework nabízí pouze sadu nástrojů a doporučení pomocí kterých zrealizujeme konečný integrační proces.

\subsection{Průběh integrace}

Průběh integrace (\textit{build}) je definován souborem \verb|master.cfg|, který je napsaný v programovacím jazyce Python \cite{buildbot_manual}.
Oproti GitLab CI a Travis CI není konfigurační soubor umístěn v repozitáři projektu, ale na Buildbot serveru.

V základu Buildbot pracuje s jednotlivými příkazy, které podle konfigurace můžeme skládat do úkolů a do fází.
Tím, že je konfigurace napsaná jako Python skript, nám Buildbot dává téměř neomezené možnosti.

V konfiguračním souboru máme přístup k třídám Buildbotu, z kterých si složíme integraci.
Základní třídou je \verb|BuildMasterConfig|, která v sobě ukládá veškeré nastavení integrace.
Příkazy shlukujeme v instanci třídy \verb|BuildFactory|, která společně s instancí třídy \verb|Worker|, která definuje běhové prostředí tvoří výsledný \verb|BuilderConfig|.
Posledním stavebním kamenem je třída \verb|Scheluder|, která je zodpovědná za naslouchání změnám v repozitáři.
Ukázka kódu \ref{code:buildbot-minimal-config} obsahuje minimální konfiguraci.

\begin{listing}[ht]
\caption{\label{code:buildbot-minimal-config}Minimální konfigurace integrace}
\begin{minted}[frame=single,linenos]{python}
from buildbot.plugins import *

c = BuildmasterConfig = {}
# Základní konfigurace
c['title'] = "Hello from Buildbot"
# ...

# Přidání Docker běhového prostředí s PHP
docker_worker = worker.DockerLatentWorker(
    "dummy_worker",
    None,
    docker_host='unix://var/run/docker.sock',
    image='php',
)
c['workers'] = [docker_worker]

# Naslouchej změnám z [GITHUB_URL] repozitáře
github_scheduler = schedulers.SingleBranchScheduler(
    name="project_name",
    change_filter=util.ChangeFilter(project='[GIT_URL]'),
    builderNames=["npm"]
)
c['schedulers'] = [github_scheduler]

# Konečná definice příkazů
# včetně stažení Git repozitáře z GitHubu
build_factory = util.BuildFactory()
build_factory.addStep(steps.GitHub(repourl='[GIT_UTL]'))
build_factory.addStep(steps.ShellCommand(command=["echo", "Hello!"]))

# Konečná definice integrace
c['builders'] = []
builder = util.BuilderConfig(
    name="php",
    workernames=["dummy_worker"],
    factory=build_factory
)

\end{minted}
\end{listing}

\subsection{Architektura}

Systém se dělí na \textit{build masters} a \textit{workers}.
\textit{Workers} jsou zodpovědní za zpracování jednotlivých příkazů definovaných v integraci.
\textit{Build master} reprezentuje samotný systém -- naslouchá změnám v repozitáři, informuje o výsledku integrací, zpřístupňuje své zdroje pomocí REST API a zadává práci podřazeným \textit{Workerům}.
Buildbot podporuje mimo Gitu i další SCM (SVN, Mercurial, \ldots) a díky modulům dokáže spolupracovat s velkým množstvím systému.
Komunikaci uvnitř systému zajišťuje Twisted framework.

\subsection{Přístup k informacím systému}

Buildbot jako dřívěji uvedená CI poskytuje přístup k informacím systému přes RESTful API.

\subsubsection{Webové rozhraní}

Webové rozhraní se aktivuje v hlavním konfiguračním souboru \verb|master.cfg|.
Jeho podobu lze dále upravovat pomocí zásuvných modulů.

\begin{listing}[ht]
\caption{\label{code:buildbot-enable-web}Povolení webového rozhraní v master.cfg}
\begin{minted}[frame=single,linenos]{python}
c['www'] = {
    'plugins': {'waterfall_view': True}
}
\end{minted}
\end{listing}

\subsection{Možnosti instalace}

Buildbot lze nainstalovat přes Python baličkovací manažer pip.
Další možností je využít připravených Docker obrazů.