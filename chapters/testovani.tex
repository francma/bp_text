\chapter{Testování}

K psaní testů je využito Python knihovny \textit{pytest} \cite{pytest} a ke správě virtuálních prostředí testů nástroj \textit{tox} \cite{python_tox}.
Úkolem nástroje \textit{tox} je vytvořit nové virtuální prostředí (\textit{virtualenv}), v něm nainstalovat závislosti projektu i projekt samotný a nakonec spustit testy dle zadání.
Použití \textit{virtuenv} prostředí odstraní možné problémy s lokálním prostředím (například různé verze knihoven v systému současně).

\begin{listing}[ht]
\begin{minted}[frame=single,linenos]{python}
def test_authorize(container: Container):
    app = container.get_app()
    app.testing = True

    user1 = User.create(
        email='user1@email.com',
        role=UserRole.NORMAL,
        public_key='AAA'
    )

    client = app.test_client()
    r = client.get('/identity', headers={
        'Authorization': 'Bearer ' + str(user1.token)
    })

    assert r.status_code is not HTTPStatus.FORBIDDEN
\end{minted}
\caption{Ukázka testu pomocí knihovny pytest}
\end{listing}

Ke statické analýze kódu je použita knihovna \textit{mypy} \cite{python_mypy}, která dokáže zkontrolovat typovou správnost z poskytnutých typových anotací.
Kontrola jednotného zápisu kódu (tzv. \uv{coding style}) byl zkontrolován oproti Python PEP specifikacím nástrojem pep8.

Testy jsou dostupné v repozitářích projektů v podsložce \verb|tests| a spouští se přes příkaz \verb|tox|.

U testování webového rozhraní byl zvolen jiný způsob testování, protože obsahuje minimální množství testovatelného kódu.
Webové rozhraní v podstatě pouze dotazuje HTTP API Piper CI jádra a interpretuje výsledky.
Pro tento účel byla vytvořena \uv{falešná} datová třída (\textit{mock}), poskytující testovací data.
Testování není tedy prováděno automaticky, ale je třeba manuálně webové rozhraní projít pro odhalení případných chyb.

\imagefigurefull{piper-ci-badges.png}{Stav pokrytí testy a CI na GitHubu}{0.33}

Při testování byla také změřena procentuální hodnota pokrytí kódu testy.
Konkrétní hodnoty jsou 87~\% pro projekt Piper CI LXD runner a 65~\% pro projekt Piper CI core.
Samotné vysoké procentuální pokrytí testy nemusí být zárukou kódu bez chyb a tak je součástí projektu i jednoduchý skript, který nainstaluje v testovacím režimu všechny části Piper CI systému.
Takto lze manuálně vyzkoušet některé mezní případy a případně je pak zahrnout v automatických testech.

V průběhu vývoje autor využil získaných zkušeností se systémy kontinuální integrace a začlenil všechny projekty vytvořené v rámci této práce do Travis CI.

\imagefigurelarge{piper-ci-travis.png}{Integrace Piper CI LXD runner na Travis CI}