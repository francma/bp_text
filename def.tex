\chyph
\input opmac
\input opmac-bib
\fontfam[Linux Libertine]
\typosize[12/18]

\def\maketitle{
	{
		\nopagenumbers
		{
			\typoscale[1400/1600]\caps\rm
			\hbox to \hsize {
				\vbox {
					\hbox{\university}
					\hbox{\faculty}
					\hbox{\department}
					\vskip 2.8pt % corecture for blank space under the LOGO
				}
				\hfill
				\vbox {
					\picheight=73.3pt \inspic img/LogoCVUT-BW.pdf
				}
			}
		}

		\vfill
		{\noindent\typoscale[1300/1450] Bakalářská práce}
		\bigskip\bigskip\bigskip
		{\noindent\typoscale[2000/2200] \title}
		\bigskip

		{\noindent\typoscale[1400/1600]\it \author}

		\vfill
		{
			{\noindent Vedoucí: \supervisor}
			\bigskip
			{\noindent\today}
		}
		\vfil
		\break
	}
	\pageno=1
}

\def\makekeywords{
	{
		\nopagenumbers
		\null
		\vfill
		{\noindent\bf Klíčová slova:}
		
		\noindent\keywords
		\vfil
		\break
	}
	\pageno=1
}

\def\today{\number\day. \number\month. \number\year}

\newcount\lnum
\def\thelnum{\thechapnum.\the\lnum}
\sdef{mt:l:en}{Listing} \sdef{mt:l:cs}{Výpis} \sdef{mt:l:sk}{Výpis}
\def\chaphook#1\relax{{\globaldefs=1 \chapnums}}
\def\chapnums{\secnum=0 \seccnum=0 \tnum=0 \fnum=0 \dnum=0 \lnum=0 }

\newcount\secccnum

\eoldef\seccc#1{%
   \ifx\prevseccnum\theseccnum \global\advance\secccnum by1
   \else \global\let\prevseccnum=\theseccnum \global\secccnum=1
   \fi
   \edef\thesecccnum{\theseccnum.\the\secccnum}%
   \printseccc{#1}%
}
\def\printseccc#1{\norempenalty-100 \medskip
   {\bf \noindent \thesecccnum\quad #1\nbpar}%
   \nobreak \smallskip \firstnoindent
}
