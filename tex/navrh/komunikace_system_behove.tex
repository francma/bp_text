\sec Komuninakce s~běhovým prostředím

Cílem je vytvořit rozhraní s~VM, které přijme zadaný příkaz nebo jejich sadu a rozhraní schopno streamovat výstup z~příkazů.
Rozhraní by mělo udržovat stály stav (proměnné shellu, ...).

\secc Virtualbox

\seccc COM port pro čtení/zápis

Pomocí VirtualBox administračního rozhraní vytvoříme COM port, který nasměrujeme do souboru/socketu.
Každý spuštěný příkaz poté přesměrujeme na zvolený COM port a necháme aplikaci na rodičovském stroji číst výstupy.
Spojení je oboustranné, takže je možné předávat příkazy i dovnitř stroje, které je ale nutné parsovat a předávat systému další aplikací běžící uvnitř VM.
Tento způsob předpokládá už přihlášeného uživatele.

\seccc SSH spojení

SSH je kryptografický protokol, který umožňuje oboustrannou komunikaci mezi účastníky.
Přihlašování uživatelů do VM bez hesla lze docílit uložením jejich veřejných klíčů do souboru {\it known\_hosts}.
Soubor {\it known\_hosts} je uložen v~domovském adresáři uživatele a bude nutno zajistit jeho modifikaci buďto přes dodatečné připojení VDI obrazu disku VM nebo předpřípravou systému.

\seccc Vzdálený terminál přes COM port

Podobné SSH spojení.
Nutná editace konfigurace VM systému, aby daný COM port viděl jako terminálový.
Přihlašování uživatele a spuštění příkazů za pomoci přímého vpisu/čtení ze socketu.

\secc Docker

Příkazy lze předávat do kontenejneru přímo z~hostujícího stroje.
Výpis výsledku je přímo na stdout, který lze přesměrovat do roury a z~té dál do CI systému.

\begtt
docker start ecstatic_perlman
docker exec -i ecstatic_perlman bash -c 'ls -al' 
\endtt
\nobreak \label[docker-minimal-example]
\caption/l Ukázka Docker exekutoru
\smallskip