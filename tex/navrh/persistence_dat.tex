\sec Ukládání dat systému

Co je třeba ukládat?

\begitems
* Uživatelé a jejich nastavení
* Výsledky buildů včetně jejich logů
* Vazby mezi CI a SCM
\enditems

\noindent
Data by měla být i jednoduše seřaditelná podle různých kritérií.

\secc Souborový systém

Nejjednodušší persistentní datové uložiště, které lze strukturovat pomocí adresářů a symlinků.
Jednotlivé soubory pak uložíme v~nějakém strojově zpracovatelném formátu, například JSON nebo XML.
Z~hlediska složité filtrace a řazení nevhodné pro data nad kterými chceme provádět tyto operace.

\medskip\noindent
{\bf Vhodné pro:}
\begitems
* logy
* binární data
* velké soubory
* konfigurační soubory
\enditems

\secc In-memory datová struktura (Redis)

Klíč-hodnota struktura uložená v~operační paměti podporující následující datové typy:

\begitems
* řetězce
* seznamy
* sady
* seřazené sady
* hashe (asociativní pole)
\enditems

\noindent
Asociativní pole lze zanořovat do sebe a tím vytvářet struktury.
Funguje na principu klient-server a je tedy snadné sdílet data mezi více systémy, které běží například na různých strojích.
Operační pamět není persistentním uložištěm a tak je třeba data ukládat také na pevný disk.
To se dělá pomocí průběžných zápisových logů nebo dumpem celé struktury do souboru.
Oba způsoby jsou přímou součástí Redisu.

\medskip\noindent
{\bf Vhodné pro:}
\begitems
* dočasná data ke kterým potřebujeme rychlý přístup
* fronty
* zámky
\enditems

\secc Databáze

\medskip\noindent
{\bf Vhodné pro:}
\begitems
* filtrovaná data
* řazená data
* data se složitějšími vazbami
\enditems
