\section{Komunikace}

Komunikace bude navazována vždy ze směru od běhového klienta

\subsection{REST architektura}

Za REST kompatibilní rozhraní je považováno rozhraní splňující následující požadavky:

\begin{itemize}
	\item Model klient-server
	\item Bezstavový model
	\item Správa mezipaměťi
	\item Uniformní rozhraní
	\item Vrstvená architektura
\end{itemize}

\subsubsection{Model klient-server}

Model klient-server je styl návhru systému, který dělí jeho části na poskytovatele služeb (servery) a jejich uživatele (klienty).
Rozdělením systému na více (na sobě nezávislých) částí je docíleno vylepšené portabtability a škálovatelnosti.

\subsubsection{Bezstavový model}

Bezstavové odbavení požadavku znamená, že každý požadavek musí obsahovat všechny informace k jeho vyřízení.
Přínosem je jednodušší odbavení požadavku z hlediska serveru, který nemusí udržovat jednotlivým klientům jejich stav (\textit{session}) nebo řešit chyby na základě nevalidního stavu.
Nevýhoda spočívá v redundanci přenesených dat.

\subsubsection{Správa mezipaměti}

Součástí odpovědi od serveru může být i informace o tom, zda je možné tuto odpověď znovupoužít.
Klient pak místo posílání dalšího stejného požadavku může použít již obdrženou odpověď z mezipaměti.
Výhodou je uvolnění systémových prostředků serveru.
Na druhé straně se může stát, že odpověd v mezipaměti je již neplatná.

\subsubsection{Uniformní rozhraní}

Rozhraní serveru je navrženo obecně, bez přizpůsobení požadavkům jednoho určitého klienta.
Tímto je dosaženo zjednodušení serverové části za cenu snížení efektivity. 

\subsubsection{Vrstvená architektura}

Příjemce požadavku a jeho vykonavatel nemusí být tentýž server.
Z hlediska klienta se v komunikaci se serverem nic nemění.
Nevýhodou je zvýšení režie a latence přenosu dat.
Přínos spočívá v umožnění rozdělení serverové části na více menších systémů, které jsou snažší na správu a umožňují lepší škálovatelnost.

\subsubsection{Reprezentace dat a přístup ke zdrojům}

Základním stavebním kamenem REST rozhraní jsou zdroje.
Každá pojmenovatelná informace může být zdrojem (např. počasí dnes v Praze, \ldots), kolekce dalších zdrojů (např. počasí v hlavních městech Evropy) a další.
Zdroj je identifikován URI adresou\footnote{schéma:[//[uživatel[:heslo]@]server[:port]][/cesta][?dotaz][\#fragment]}.




% https://www.ics.uci.edu/~fielding/pubs/dissertation/rest_arch_style.htm

\subsection{HTTP}


\subsection{Websocket}

Websocket protokol umožňující oboustranout komunikaci mezi klientem a hostitelem.

% https://tools.ietf.org/html/rfc6455






\subsection{Message Based}

Implementace distribuovaných prioritních front ve stylu producenta a konzumenta.
Použití: máme více Runnerů a chceme jim na základě jejich vytížení přiřazovat úkoly.
