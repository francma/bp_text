\chap Úvod

Programování a procesy s~ním spojené prochází neustálým vývojem, který tuto disciplínu stále mění. 
Přidáváním vyšších vrstev abstrakce se vzdalujeme od stroje samotného a přecházíme k~simulaci fungování reálného světa, čímž se stává vývoj softwaru neustále rychlejším a dynamičtějším. 


Jsou to také nově vznikající vývojové techniky, které urychlují vývoj softwaru.
Jednou z~nich je i technika kontinuální integrace (zkráceně CI). Tato technika spočívá v~časté integraci kódu mezi vývojáři ve společném repozitáři a následným testováním. Mezi testy patří:

\begitems
* Integrační a jednotkové testy kontrolující funkčnost programu
* Testy kontrolující jednotný styl kódu v~projektu
* Syntaktické testy
\enditems

Kontinuální integrace je úzce spjatá s~verzovacím systémem, který spravuje repozitář projektu. Samotný systém, realizující kontinuální intgraci, lze definovat jako množinu softwaru, realizující následující úkony:

\begitems
* Naslouchat a reagovat na změny v~repozitáři projektu
* Připravit běhové prostředí vhodné pro testovací proces
* Stáhnout zdrojové soubory projektu z~repozitáře
* Sestavit stážený projekt včetně jeho závislostí
* Spustit testovaní dle zadání projektu
* Informovat vývojáře a přidružené systémy o~výsledku testování
\enditems

\input tex/uvod/struktura
\input tex/uvod/behova_prostredi
\input tex/uvod/scm