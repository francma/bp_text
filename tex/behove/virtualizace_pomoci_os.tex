\sec Virtualizace na úrovni operačního systému

% http://www.linfo.org/kernel_space.html
Systémová paměť moderních operačních systémů je dělena na dvě části: prostor jádra a uživatelský prostor.
Prostor jádra je přímo využíván pro běh jádra operačního systému a jeho služeb.
V~uživatelském prostoru běží procesy spuštěné uživatelem bez přístupu do prostoru jádra.
Uživatelské procesy komunikují s~prostorem jádra pouze skrz systémové metody.
Na x86 procesorech se k~rozdělení paměti používá chráněný režim, ve kterém lze stránky virtuální paměti rozdělit podle privilegovanosti.

Virtualizace na úrovni operačního systému je virtualizační technika, která dovoluje koexistenci několika oddělených uživatelských režimů, které fungují nad stejným prostorem jádra.
Izolovanému uživatelskému prostoru se často označuje jako tzv. \uv{kontejner}.

Výhodou je menší režie oproti plné virtualizaci, protože už není nutné spouštět celou novou instanci operačního systému.
Oproti plné virtualizaci je naopak nutné zajistit, aby se jednotlivé uživatelské režimy nemohly neoprávněně ovlivňovat.
Toto je netriviální úloha a proto je vhodné zavést některá bezpečností opatření.
Jedním z~opatření je vytvoření kontejneru pod neprivilegovaným uživatelem (ne-root, ne-administrátor), takže případní \uv{únik} z~kontejneru nezpůsobí velkou nebo žádnou škodu.
Dalším opatřením může být spuštění nové instance operačního systému pomocí plné virtualizace, která bude využívána pro běh kontejnerů.
Únikem z~kontejneru se pak útočník dostane pouze do virtualizovaného operačního systému bez možnosti ovlivnit hostitele.

{
\smallskip\noindent
Mezi technologie pro tvorbu kontejnerů patří:

\begitems
* Jails (BSD)
* Namespaces a Cgroups (Linux)
* OpenVZ
* Windows kontejnery
* Multiplaformní Docker
\enditems
}

\secc Linux

\seccc Namespaces

Nástroj Linux kernelu sloužící k~vytvoření izolovaných skupin procesů.
Každý proces má svůj namespace, který mu vymezuje viditelnost pouze na prostředky daného namespace nebo jeho podřezených.
Při startu systému existuje nejméně jeden namespace.
Namespaci tvoří hiearchickou stromovou strukturu.
Existuje několik druhů namespaců.

\smallskip\noindent{\bf PID}\break\noindent
Proces vidí procesy pouze ze své PID namespace a namespaců podřízených.

\smallskip\noindent{\bf NET}\break\noindent
Síťové rozhraní náleží právě do jednoho namespacu a není sdíleno s~namespaci podřízenými.
Každé síťové rozhraní má svoje IP adresy, routovací tabulky, firewall a další síťové prostředky. 

\smallskip\noindent{\bf MNT}\break\noindent
Kontrolujeme jaké mount pointy budou viditelné.
Vytvořením nového MNT namespacu přeneseme všechny mount pointy rodiče.
Změny v~aktuálním namespacu neovlivňují rodičovský namespace. 

\smallskip\noindent{\bf IPC}\break\noindent
Meziprocesová komunikace.
Linux kernel ve výchozím stavu přiděluje sdílenou paměť na základě uživatelského ID, což by vedlo ke kolizím mezi namespacy.

% FIXME? PostreSQL pokud má stejné UID, pak přistupuje do stejné paměti a koliduje??

\smallskip\noindent{\bf UTS}\break\noindent
Dovoluje systému vystupovat pod více hostnames.

\smallskip\noindent{\bf user}\break\noindent
Umožňuje libovolně mapovat uživatelská ID z~kontejneru na ID uživatelů z~hostitele.
Uživatel v~kontejneru se pak může tvářit jako root i když vně kontejneru se jedná pouze o~obyčejného uživatele (bez práv roota).

% https://www.youtube.com/watch?v=sK5i-N34im8

O~vytvoření namespacu se stará systémové volání FORK.

\seccc Control groups (cgroups)

Nástroj Linux kernelu umožňující přidělovat skupinám procesů limit na systémové prostředky a prioritizovat jednotlivé skupiny při přístupu k~nim.
Typy prostředků jsou:
\begitems
* CPU čas
* Operační paměť
* Disková paměť % FIXME tohle je divny nazev
* Síťový provoz
* Provoz na disku
\enditems

Mezi další vlastnosti Control Groups patří možnost zmrazit skupinu procesů v~určitém stavu a následně je znovu odmrazit. % FIXME mam tohle vubec rozebirat


\seccc Copy on write

Technika optimalizace sprvávy dat, při které se při kopírování dat namísto vytvoření jejich kopie pouze označí jako sdílená.
Fyzickou kopii dat je třeba vytvořit až v~okamžiku jejich modifikace.
Za cenu menší režie můžeme takto ušetřit mnoho paměti.

% https://books.google.cz/books?id=9yIEji1UheIC&lpg=PA295&dq="copy+on+write"&pg=PA295&redir_esc=y#v=onepage&q="copy%20on%20write"&f=false

Použití v~Linuxu při volání FORK nebo v~různých filesystémech.

\secc BSD jails

% https://www.freebsd.org/doc/handbook/jails.html

\secc Docker

\secc Windows kontejnery
