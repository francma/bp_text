% arara: xelatex: { shell: yes }
% arara: makeglossaries
% arara: biber
% arara: xelatex: { shell: yes }
% arara: xelatex: { shell: yes }
% arara: xelatex: { shell: yes }

\documentclass[thesis=B,czech]{template/FITthesisXE}
\usepackage{graphicx}
\usepackage{dirtree}
\usepackage{xevlna}
\usepackage{caption}
\usepackage{minted}
\usepackage{rotating}
\usepackage{anyfontsize}
\usepackage{listings}
\usepackage[export]{adjustbox}

\newcommand{\uv}[1]{\quotedblbase #1\textquotedblleft}

\bibliography{library.bib}

% usage: \imagefigure{filename}{description}
\newcommand{\imagefigurefull}[3]{
    \begin{figure}[htbp]
    \centering
        \includegraphics[width=#3\linewidth]{media/#1}
        \caption{#2 \label{pic:#1}}
    \end{figure}
}


\newcommand{\imagefigurelarge}[2]{
    \imagefigurefull{#1}{#2}{.99}
}


\newcommand{\imagefigure}[2]{
    \imagefigurefull{#1}{#2}{.6}
}

\makeglossaries
\newacronym{CI}{CI}{Kontinuální integrace}
\glsaddall	% add even unused acronyms


\department{Katedra počítačových systémů}
\title{Minimalistický CI systém}
\authorGN{Martin}
\authorFN{Franc}
\authorWithDegrees{Martin Franc}
\author{Martin Franc}
\supervisor{Ing. Jakub Jirůtka}
% \acknowledgements{Doplňte, máte-li komu a za co děkovat. V~opačném případě úplně odstraňte tento příkaz.}
\abstractEN{\input abstract_en.tex}
\abstractCS{\input abstract_cs.tex}
\placeForDeclarationOfAuthenticity{V~Praze}
\declarationOfAuthenticityOption{4}
\keywordsCS{CI, kontinuální integrace, virtualizace, LXD}
\keywordsEN{CI, continuous integration, virtualization, LXD}
\assignment{assignment.pdf}


\begin{document}

\input chapters/uvod.tex
\chapter{Úvod do problematiky}

Tento popis je velmi obecný a proto bude vhodné se podívat na nějaký konkrétní projekt používající kontinuální integraci.
Vezmeme například projekt Nette, který umožňuje stavbu webových aplikací.
Repozitáře projektu jsou verzované verzovacím nástrojem Git a další integrace je pod systémem kontinuální integrace Travis CI.

\begin{enumerate}
	\item Zkontroluj syntakticky správný zápis programu.
	\item Spusť integrační a jednotkové testy pro verze PHP 7.1 a 7.2 pro terminálové rozhraní (php-cgi) a webové (php).
	\item Předej informace o testovaní službě třetí strany.
\end{enumerate}

Z tohoto nám vznikne 6 úkolů:

\begin{itemize}
	\item Kontrola syntaxe.
	\item Testy na php 7.1.
	\item Testy na php 7.2.
	\item Testy na php-cgi 7.1.
	\item Testy na php-cgi 7.2.
	\item Předání informací o testování službě třetí strany.
\end{itemize}

Jelikož nemá smysl pouštět jakékoliv testování, pokud program není zapsán syntakticky správně, tak jako první provedeme kontrolu syntaxe.
Po zkontrolovaní syntaktické správnosti spustíme integrační a jednotkové testy.
Protože testy na sobě nijak nezávisí, tak je možné je pustit paralelně, abychom zkrátili dobu testování.
Po dokončení testování předáme výsledky službě třetí strany.

Skupiny testů, které jsou na sobě nezávislé a je možné je paralelizovat nazýváme fází.
V tomto konkrétním případě nám vzniknou 3 fáze:

\begin{itemize}
	\item Kontrola syntaxe.
	\item Testy.
	\item Předání informací o testování službě třetí strany.
\end{itemize}

Od systému kontinuální integrace tedy očekáváme určitou možnost integrační proces řídit.
Většinou se jedná o rozdělení na fáze a úkoly, ale také o další možnosti, které budou probrány u konkrétních systémů CI.

Uživatele CI systému také zajímá výstup z jeho zadaných příkazů.
Velmi vhodné je to například při hledání chyby, kdy by pouhá informace o neúspěchu integrace byla nedostačující.
Mnoho CI systémů implementuje tzv. \uv{streaming logy}, což znamená, že výstup z příkazů je dostupný v \uv{reálném čase} místo nutnosti čekat na doběhnutí celé integrace.

S technikou kontinuální integrace souvisí i technika kontinuálního nasazení (zkráceně CD).
Tato technika vychází z toho, že pokud je projekt otestovaný a sestavený, tak jej je možné nasadit do produkce.
Mnoho systémů na kontinuální integraci v sobě kombinují i kontinuální nasazení.

\input chapters/sw-testovani.tex
\input chapters/scm.tex
\input chapters/behove.tex
\chapter{Analýza existujících CI řešení}
\input chapters/gitlab.tex
\input chapters/travis.tex
\input chapters/buildbot.tex
\input chapters/navrh.tex
\input chapters/implementace.tex
\input chapters/testovani.tex
\input chapters/zaver.tex

\printbibliography[]


\appendix

\chapter{Seznam zkratek}
\printglossary[type=\acronymtype,style=acronyms]

\chapter{Seznam příloh}

Kompletní zdrojový kód této práce a projektů vytvořených v této práci je k nalezení na přiloženém médiu.

Obsah je také dostupný online na GitHubu. 
Odevzdané verze jsou označené \texttt{thesis}.

\noindent \textbf{Tato práce} \hfill
\url{https://github.com/francma/bp_text}

\noindent \textbf{Piper CI meta} \hfill
\url{https://github.com/francma/piper-ci}

\noindent \textbf{Piper CI core} \hfill
\url{https://github.com/francma/piper-ci-core}
    
\noindent \textbf{Piper CI web} \hfill
\url{https://github.com/francma/piper-ci-web}

\noindent \textbf{Piper CI lxd} \hfill
\url{https://github.com/francma/piper-ci-lxd-runner}

\vfill

\begin{dirfigure}%
    \dirtree{%
        .1 .
        .1 README.md        \DTcomment{krátký popis k obsahu média}.
        .1 BT\_Franc\_Martin\_CAS\_COMMIT.pdf \DTcomment{práce v PDF formátu}.
        .1 bp\_text/          \DTcomment{zdrojové kódy práce v \XeLaTeX{}}.
        .1 piper-ci/  \DTcomment{repozitář meta projektu Piper CI}.
        .1 piper-ci-core/  \DTcomment{repozitář Piper CI core}.
        .1 piper-ci-web/  \DTcomment{repozitář Piper CI web}.
        .1 piper-ci-lxd-runner/  \DTcomment{repozitář Piper CI lxd}.
    }
\caption{Obsah přiloženého média}
\end{dirfigure}
\chapter{Dokumentace}

\section{REST API}



\begin{sidewaystable}[h]
\fontsize{8.3}{10}\selectfont
\centering
\begin{tabular}{|l|l|}
\hline
\verb|GET    /|                                          & Vrací schéma API ve formátu OpenAPI \\ \hline
\verb|GET    /builds|                                    & Vrací seznam integrací \\ \hline
\verb|GET    /builds/[build-id]|                         & Vrací integraci s identifikátorem \verb|[build-id]| \\ \hline
\verb|POST   /builds/[build-id]/cancel|                  & Přeruší probíhající integraci s identifikátorem \verb|[build-id]| \\ \hline
\verb|POST   /builds/[build-id]/cancel|                  & Restartuje integraci s identifikátorem \verb|[build-id]| \\ \hline
\verb|GET    /builds/[build-id]/stages|                  & Vrací seznam fází integrace s identifikátorem \verb|[build-id]| \\ \hline
\verb|GET    /jobs|                                      & Vrací seznam úkolů \\ \hline
\verb|GET    /jobs/[job-id]|                             & Vrací úkol s identifikátorem \verb|[job-id]| \\ \hline
\verb|POST   /jobs/[job-id]/cancel|                      & Přeruší probíhající úkol s identifikátorem \verb|[job-id]| \\ \hline
\verb|POST   /jobs/[job-id]/cancel|                      & Restartuje úkol s identifikátorem \verb|[job-id]| \\ \hline
\verb|GET    /jobs/queue/[runner-token]|                 & Vrací úkol ke zpracování běhovým klientem identifikovaným \verb|[runner-token]| \\ \hline
\verb|POST   /jobs/report/[job-secret]|                  & Přijímá data od běhového klienta o probíhajícím úkolu, obsahuje stav úkolu a log \\ \hline
\verb|GET    /jobs/[job-id]/log|                         & Vrací log úkolu \\ \hline
\verb|GET    /projects|                                  & Vrací seznam projektů \\ \hline
\verb|POST   /projects|                                  & Vytváří nový projekt \\ \hline
\verb|GET    /projects/[project-id]|                     & Vrací projekt s identifikátorem \verb|[project-id]| \\ \hline
\verb|PUT    /projects/[project-id]|                     & Editace projektu s identifikátorem \verb|[project-id]| \\ \hline
\verb|DELETE /projects/[project-id]|                     & Smazání projektu s identifikátorem \verb|[project-id]| \\ \hline
\verb|GET    /projects/[project-id]/builds|              & Vrací seznam integrací spojených s projektem identikovaným \verb|[project-id]| \\ \hline
\verb|GET    /projects/[project-id]/stages|              & Vrací seznam fází spojených s projektem identikovaným \verb|[project-id]| \\ \hline
\verb|GET    /projects/[project-id]/users|               & Vrací seznam uživatelů přiřazených do projektu identikovaným \verb|[project-id]| \\ \hline
\verb|POST   /projects/[project-id]/users|               & Přidává uživatele do projektu identikovaným \verb|[project-id]| \\ \hline
\verb|DELETE /projects/[project-id]/users/[user-id]|     & Odstraňuje uživatele z projektu identikovaným \verb|[project-id]| \\ \hline
\verb|GET    /stages|                                    & Vrací seznam fází \\ \hline
\verb|GET    /stages/[stage-id]|                         & Vrací fázi s identifikátorem \verb|[stage-id]| \\ \hline
\verb|POST   /stages/[stage-id]/cancel|                  & Přeruší probíhající fázi s identifikátorem \verb|[stage-id]| \\ \hline
\verb|POST   /stages/[stage-id]/cancel|                  & Restartuje fázi s identifikátorem \verb|[stage-id]| \\ \hline
\verb|GET    /runners|                                   & Vrací seznam běhových klientů \\ \hline
\verb|POST   /runners|                                   & Vytváří nového běhového klienta \\ \hline
\verb|GET    /runners/[runner-id]|                       & Vrací běhového klienta s identifikátorem \verb|[runner-id]| \\ \hline
\verb|PUT    /runners/[runner-id]|                       & Edituje běhového klienta s identifikátorem \verb|[runner-id]| \\ \hline
\verb|DELETE /runners/[runner-id]|                       & Maže běhového klienta s identifikátorem \verb|[runner-id]| \\ \hline
\verb|GET    /users|                                     & Vrací seznam uživatelů \\ \hline
\verb|POST   /users|                                     & Vytváří nového uživatele \\ \hline
\verb|GET    /users/[user-id]|                           & Vrací uživatele s identifikátorem \verb|[user-id]| \\ \hline
\verb|PUT    /users/[user-id]|                           & Edituje uživatele s identifikátorem \verb|[user-id]| \\ \hline
\verb|DELETE /users/[user-id]|                           & Maže uživatele s identifikátorem \verb|[user-id]| \\ \hline
\verb|GET    /identity|                                  & Vrací aktuální identitu na základě autentizačního tokenu \\ \hline
\end{tabular}
\end{sidewaystable}

\clearpage\section{Terminálové rozhraní}
{

\setminted{fontsize=\scriptsize,baselinestretch=1}

\begin{listing}[H]
\begin{minted}[frame=single]{text}
$ help

Documented commands (type help <topic>):
========================================
build  exit  help  identity  job  project  runner  stage  user
\end{minted}
\end{listing}

\begin{listing}[H]
\begin{minted}[frame=single]{text}
$ help identity

> identity get
> identity update [email = str] [role = (master, admin, normal)] 
                  [token = str] [public_key = str]
\end{minted}
\end{listing}

\begin{listing}[H]
\begin{minted}[frame=single]{text}
$ help identity

> identity get
> identity update [email = str] [role = (master, admin, normal)] 
                  [token = str] [public_key = str]
\end{minted}
\end{listing}

\begin{listing}[H]
\begin{minted}[frame=single]{text}
$ help project

> project get [project_id]
> project list [url = str] [origin = str]
  [limit = int] [offset = int] [order = (created-desc|created-asc)]
> project count [url = str] [origin = str]
> project update [project_id] [url = str] [origin = str]
> project create [url = str] [origin = str]
\end{minted}
\end{listing}

\begin{listing}[H]
\begin{minted}[frame=single]{text}
$ help build

> build get [build_id]
> build list [project_id = int] [branch = str]
  [status = (created|ready|pending|running|failed|success|canceled|skipped|error)]
  [limit = int] [offset = int] [order = (created-desc|created-asc)]
> build count [project_id = int] [branch = str] [status = ()]
  [limit = int] [offset = int] [order = (created-desc|created-asc)]
> build cancel [build_id]
> build restart [build_id]
\end{minted}
\end{listing}

\begin{listing}[H]
\begin{minted}[frame=single]{text}
$ help stage

> stage get [stage_id]
> stage list [project_id = int] [build_id = int]
  [status = (created|ready|pending|running|failed|success|canceled|skipped|error)]
  [limit = int] [offset = int]
> stage count
  [project_id = int] [build_id = int]
  [status = (created|ready|pending|running|failed|success|canceled|skipped|error)]
> stage cancel [stage_id]
> stage restart [stage_id]
\end{minted}
\end{listing}

\begin{listing}[H]
\begin{minted}[frame=single]{text}
$ help job

> job get [job_id]
> job list
  [project_id = int] [build_id = int] [stage_id = int]
  [status = (created|ready|pending|running|failed|success|canceled|skipped|error)]
  [limit = int] [offset = int]
> job count
  [project_id = int] [build_id = int] [stage_id = int]
  [status = (created|ready|pending|running|failed|success|canceled|skipped|error)]
> job cancel [job_id]
> job restart [job_id]
> job log [job_id]
\end{minted}
\end{listing}

\begin{listing}[H]
\begin{minted}[frame=single]{text}
$ help user

> user get [user_id]
> user list [email = str]
  [limit = int] [offset = int] [order = (created-desc|created-asc)]
> user count [email = str]
  [limit = int] [offset = int] [order = (created-desc|created-asc)]
> user create [email = str] [role = (master, admin, normal)] [token = str]
  [public_key = str]
> user update [email = str] [role = (master, admin, normal)] [token = str]
  [public_key = str]
> user delete [user_id]
\end{minted}
\end{listing}

\begin{listing}[H]
\begin{minted}[frame=single]{text}
$ help runner

> runner get [runner_id]
> runner list [group = str]
  [limit = int] [offset = int]
> runner count [group = str]
> runner create [group = str] [token = str]
> runner update [group = str] [token = str]
> runner delete [runner_id]
\end{minted}
\end{listing}
}

\clearpage\section{Postup nasazení}

\subsection{Piper CI core}

Požadavky na systém:

\begin{itemize}
	\item OS Linux,
	\item Redis server,
	\item Python 3.6+,
	\item SSH server,
	\item Git.
\end{itemize}

Doporučené požadavky:

\begin{itemize}
	\item SQL server (MariaDB, PostgreSQL, \ldots)
	\item WSGI server (uwsgi, gunicorn, \ldots)
	\item Reverzní proxy (nginx, apache, \ldots)
\end{itemize}

\subsection{Piper CI LXD runner}

Požadavky na systém:

\begin{itemize}
	\item OS Linux,
	\item Python 3.6+,
	\item Git,
	\item LXD.
\end{itemize}

\subsection{Piper CI web}

Požadavky na systém:

\begin{itemize}
	\item OS Linux,
	\item Python 3.6+.
\end{itemize}

Doporučené požadavky:

\begin{itemize}
	\item WSGI server (uwsgi, gunicorn, \ldots)
	\item Reverzní proxy (nginx, apache, \ldots)
\end{itemize}

Instalace:

\begin{enumerate}
	\item Nainstalujeme balíček pomocí pip: \newline
		\verb|pip install git+https://github.com/francma/piper-ci-web.git|
	\item Vytvoříme produkční konfiguraci: \newline
		\verb|cp config.example.yml config.yml|.
	\item Konfiguraci upravíme podle instrukcí v souboru \verb|config.yml|
\end{enumerate}

Spuštění:
\begin{listing}[H]
\begin{minted}[frame=single]{text}
uwsgi --http-socket :[PORT] 
      -w piper_web.run:app
      --pyargv [CONFIG_FILE]
\end{minted}
\end{listing}

\end{document}
