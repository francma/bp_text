% arara: xelatex: { shell: yes }
% arara: makeglossaries
% arara: biber
% arara: xelatex: { shell: yes }
% arara: xelatex: { shell: yes }

\documentclass[thesis=B,czech]{FITthesisXE}
\usepackage{graphicx}
\usepackage{dirtree}
\usepackage{xevlna}
\usepackage{caption}
\usepackage{minted}
\usepackage[export]{adjustbox}

\newcommand{\tg}{\mathop{\mathrm{tg}}}
\newcommand{\cotg}{\mathop{\mathrm{cotg}}}
\newcommand{\uv}[1]{\quotedblbase #1\textquotedblleft}
\renewcommand{\thefootnote}{\arabic{footnote}}

\bibliography{library.bib}

% usage: \imagefigure{filename}{description}
\newcommand{\imagefigurefull}[3]{
    \begin{figure}[htbp]
    \centering
        \includegraphics[width=#3\linewidth]{media/#1}
        \caption{#2 \label{pic:#1}}
    \end{figure}
}

\newcommand{\imagefigurelarge}[2]{
    \imagefigurefull{#1}{#2}{.99}
}

\newcommand{\imagefigure}[2]{
    \imagefigurefull{#1}{#2}{.6}
}

\makeglossaries
\newacronym{CI}{CI}{Kontinuální integrace}
\newacronym{OS}{OS}{Operační systém}
\newacronym{CPU}{CPU}{Central Processing Unit}
\newacronym{API}{API}{Application programming interface}
\newacronym{COM}{COM}{Communication port}
\newacronym{CoW}{CoW}{Copy on Write}
\newacronym{I/O}{I/O}{Input/Output}
\newacronym{ANSI}{ANSI}{American National Standards Institute}
\newacronym{BSD}{BSD}{Berkeley Software Distribution}
\newacronym{CLI}{CLI}{Command-line interface}
\newacronym{DHCP}{DHCP}{Dynamic Host Configuration Protocol}
\newacronym{HTML}{HTML}{HyperText Markup Language}
\newacronym{HTTP}{HTTP}{Hypertext Transfer Protocol}
\newacronym{ICMP}{ICMP}{Internet Control Message Protocol}
\newacronym{IP}{IP}{Interner Protocol}
\newacronym{JSON}{JSON}{JavaScript Object Notation}
\newacronym{KVM}{KVM}{Kernel-based Virtual Machine}
\newacronym{LXC}{LXD}{Linux Containers}
\newacronym{PEP}{PEP}{Python Enhancement Proposals}
\newacronym{RAM}{RAM}{Random-access memory}
\newacronym{SDK}{SDK}{Software development kit}
\newacronym{SQL}{SQL}{Structured Query Language}
\newacronym{SSH}{SSH}{Secure Shell}
\newacronym{UI}{UI}{User interface}
\newacronym{URI}{URI}{Uniform Resource Identifier}
\newacronym{URL}{URL}{Uniform Resource Locator}
\newacronym{VCS}{VCS}{Version control}
\newacronym{VM}{VM}{Virtual Machine}
\newacronym{XML}{XML}{Extensible Markup Language}
\newacronym{YAML}{YAML}{YAML Ain't Markup Language}
\glsaddall	% add even unused acronyms


\department{Katedra počítačových systémů}
\title{Minimalistický CI systém}
\authorGN{Martin}
\authorFN{Franc}
\authorWithDegrees{Martin Franc}
\author{Martin Franc}
\supervisor{Jakub Jirůtka}
% \acknowledgements{Doplňte, máte-li komu a za co děkovat. V~opačném případě úplně odstraňte tento příkaz.}
\abstractEN{\input abstract_en.tex}
\abstractCS{\input abstract_cs.tex}
\placeForDeclarationOfAuthenticity{V~Praze}
\declarationOfAuthenticityOption{4}
\keywordsCS{Nahraďte seznamem klíčových slov v češtině oddělených čárkou.}
\keywordsEN{Nahraďte seznamem klíčových slov v angličtině oddělených čárkou.}
\assignment{assignment.pdf}


\begin{document}

\input chapters/0-uvod.tex
\input chapters/1-behove.tex
\input chapters/2-scm.tex
\input chapters/3-gitlab.tex
\input chapters/4-travis.tex
\input chapters/5-buildbot.tex
\input chapters/6-navrh.tex
\input chapters/7-implementace.tex
\input chapters/8-testovani.tex
\input chapters/9-zaver.tex

\printbibliography[]


\appendix

\chapter{Seznam zkratek}
\printglossary[type=\acronymtype,style=acronyms]

\chapter{Seznam příloh}

Kompletní zdrojový kód této práce a projektů vytvořených v této práci je k nalezení na přiloženém médiu.

Obsah je také dostupný online na GitHubu. 
Odevzdané verze jsou označené \texttt{thesis}.

\noindent \textbf{Tato práce} \hfill
\url{https://github.com/francma/bp_text}

\noindent \textbf{Piper CI meta} \hfill
\url{https://github.com/francma/piper-ci}

\noindent \textbf{Piper CI core} \hfill
\url{https://github.com/francma/piper-ci-core}
    
\noindent \textbf{Piper CI web} \hfill
\url{https://github.com/francma/piper-ci-web}

\noindent \textbf{Piper CI lxd} \hfill
\url{https://github.com/francma/piper-ci-lxd-runner}

\vfill

\begin{dirfigure}%
    \dirtree{%
        .1 .
        .1 README.md        \DTcomment{krátký popis k obsahu média}.
        .1 BP\_Franc\_Martin.pdf \DTcomment{práce v PDF formátu}.
        .1 bp\_text/          \DTcomment{zdrojové kódy práce v \XeLaTeX{}}.
        .1 piper-ci/  \DTcomment{repozitář meta projektu Piper CI}.
        .1 piper-ci-core/  \DTcomment{repozitář Piper CI core}.
        .1 piper-ci-web/  \DTcomment{repozitář Piper CI web}.
        .1 piper-ci-lxd-runner/  \DTcomment{repozitář Piper CI lxd}.
    }
\caption{Obsah přiloženého média}
\end{dirfigure}

\end{document}
